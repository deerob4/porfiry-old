\subsection{Justification of Methods} % (fold)
\label{sub:justification_of_methods}
As is custom in such situations, a range of investigatory methods have been used to gain a deep insight into the current methods used by the school in their management of quizzes. By using a range of methods, including interviews, document inspections and observations, it is possible to uncover a wide range of information relating to the topic, all of which can be used to develop a suitable range of objectives that result in a system that meets the school's requirements.

\subsubsection{Interviews} % (fold)
\label{ssub:interviews}
The most useful method of investigation used was interviews. Specific members of staff were chosen for interview, ranging from normal teachers, to more senior figures, to the headmaster himself. By doing this, it was possible to gain an appreciation for how staff would like an automated quiz system to work, and how it could fit into the school structure. Interviews are particularly suited to this sort of task because they allow for a real conversation to be developed, as can be seen in the above transcripts. The interviewee can be asked to expand on interesting points, ensuring that as much information as possible is gleaned from the session. Of course, some members of staff would be uncomfortable with such a direct meeting (some use teaching styles that are best described as indirect), but all of the staff chosen for these interviews were perfectly willing to speak, and a number of spoke candidly and honestly about their opinions regarding the management of the school and how this could impact the success of the project.
% subsubsection interviews (end)

\subsubsection{Questionnaires} % (fold)
\label{ssub:questionnaires}
Though questionnaires are very useful and can provide a wide range of data, it was felt that, for this investigation, they would not be appropriate. Part of this comes down to the fact that other methods investigation are enough: interviews, document inspections and observations all provide a wealth of information, and there is little that a questionnaire could provide that the other methods could not. The second issue comes down to the policies in place within the school regarding questionnaires from external parties. Though the cause is a good one, and the questionnaire would originate from a trusted source, the school has a policy of declining requests for questionnaires to be given out, on account of the possibility that students could be exposed to ideas not in keeping with those officially sanctioned by the leadership team.
% subsubsection questionnaires (end) subsection justification_of_methods (end)
