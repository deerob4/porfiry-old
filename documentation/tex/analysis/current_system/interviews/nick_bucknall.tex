\textit{\textbf{DR:}} Okay. So, for my Computing coursework, I've got to build a system, for a business. I've chosen here, because I like it here.

\textit{\textbf{NB:}} Okay. \textit{\textbf{*sniggers*}}

\textit{\textbf{DR:}} I thought a good idea for the system would be something that helps improve form times...

\textit{\textbf{NB:}} Okay...

\textit{\textbf{DR:}} And so I thought about a quiz system. I remember that last year quizzes were something that you did quite often.

\textit{\textbf{NB:}} Yep.

\textit{\textbf{DR:}} I thought that the way they were handled could have been improved.

\textit{\textbf{NB:}} Okay. Are you aware that the structure's changing?

\textit{\textbf{DR:}} Yes, with the heads of house.

\textit{\textbf{NB:}} Yes. Alright. So we were horizontal - obviously in year groups, as you know - now we're going vertical, and we're in houses. We're running a weekly quiz, that's probably going to fit in... I say weekly, but it's probably going to be three times a half term. At the moment, we're asking the house captains to make the quiz, and then they're going to dish it out, and it's supposed to be relevant for years 7 - 11. They're going to dish it out, and it's going to be done, and then marked. The problem with that is trying to make it relevant to all year groups, and, if we try and do it on the computer, not all classrooms have access to computer rooms, and...but we are obviously doing the iPads thing now. So Years 8 and 9 do have iPads, so having it on a...an app would be really useful. So what's your plan?

\textit{\textbf{DR:}} Well the idea was - and I've spoken to Mr Warr and Miss Blower about this, and they seemed to like it - part of the app would let you make the quizzes, and then the different houses would be able to find the quiz on the app, answer it, and then have the quiz marked automatically.

\textit{\textbf{NB:}} Okay, good.

\textit{\textbf{DR:}} It would then produce graphs and such. What are your initial thoughts on that?

\textit{\textbf{NB:}} Yeah, sounds alright.

\textit{\textbf{DR:}} Right. An extension of that idea, that I spoke to Miss Blower about, and she seemed really pleased with, was if there was an event, each term, where the school would get together, and each student would have an iPad, or a computer. They'd login to the system, and it would be like a live quiz. So, say the question is ``how many wives did Henry VIII have?'', that would come up on all the system's at once, and then each student would have their answer.

\textit{\textbf{NB:}} So would it be on a timer?

\textit{\textbf{DR:}} Yes.

\textit{\textbf{NB:}} What if, just working in a school environment, where things go wrong, or somebody knocks on the door and you have to speak to somebody, what would happen if there was an interruption?

\textit{\textbf{DR:}} How do you mean an interruption?

\textit{\textbf{NB:}} So if you've got all of...everybody's on their iPad, let me see if I'm understanding you right. Everybody's on their iPad, they're all in different rooms, and it automatically is generated on everybody's iPad at the same time, the question? Yep. Then if you have a certain set time, is it going to be when you've put your answer in it generates the next question, or is it going to be on a timer for the whole school?

\textit{\textbf{DR:}} Yes.

\textit{\textbf{NB:}} Okay. So my question is...if it's on a timer, you're on question 1, and somebody knocks on your door, and says ``sorry sir, so and so's just arrived in reception, can you send John down because his mum's here to pick him up'', and I'm like ``yeah okay, John go out'', and I look down and now we're on question 4, because we've missed out question 2 and 3 because I've been distracted, or the class got distracted. Could that happen?

\textit{\textbf{DR:}} Potentially, yes.

\textit{\textbf{NB:}} Okay, so you'd have to really try and minimise interruptions. While the quiz was happening you'd have to almost treat it like exam conditions.

\textit{\textbf{DR:}} Sort of. I suppose so.

\textit{\textbf{NB:}} Over the whole school.

\textit{\textbf{DR:}} I see where you're coming from.

\textit{\textbf{NB:}} I'm just trying to be awkward...

\textit{\textbf{DR:}} I'm sure.

\textit{\textbf{NB:}} ...because when we do these things, it sounds good, but because it's a school environment, things do pop up, things do happen. Somebody passes out in the classroom, and then you have to go and fix them, and then everybody's flustered, and then you look back and then when you work on a timer, it's sometimes, it's tricky to stick to. It can be. It probably would work 8 or 9 times out of ten. I mean, I've only had two kids pass out in my classroom...

\textit{\textbf{DR:}} That's pretty good!

\textit{\textbf{NB:}} ...this year.

\textit{\textbf{DR:}} Oh. Still.

\textit{\textbf{NB:}} Yeah. That's the thing - it does happen, and that would spoil something that was done on a timer. How long start to finish would you envisage the quiz lasting?

\textit{\textbf{DR:}} Well, I thought about 15 minutes. But...

\textit{\textbf{NB:}} That's quite a long time.

\textit{\textbf{DR:}} Yeah. But Miss Blower wanted half an hour, split up into rounds.

\textit{\textbf{NB:}} Right. See the rounds...the rounds thing would work if you could have a pause and a break. But then everybody would start off at the same time again?

\textit{\textbf{DR:}} Yes.

\textit{\textbf{NB:}} Okay. So you could get your kind of...anything that...somebody knocked on your door you could be like ``need two minutes''. Then you'd have a breather time. Yeah, I think breaking it up would be more manageable for a school environment, just in case something did happen.

\textit{\textbf{DR:}} So, if say the most popular answer in 8A was 4 wives, then that would become 8A's answer...\\

\begin{center}
\textit{\textbf{*NB gives out an appreciative ``ohhhh'' indicating that he has finally grasped the concept.*\\}}
\end{center}

\textit{\textbf{\\DR:}} ...and if that's then spread through the house, so say 8A, 9A and 10A chose 4 wives, then that would become Acton's answer.

\textit{\textbf{NB:}} Okay. I like that. What if they all choose something different?

\textit{\textbf{DR:}} I'd have to think about that.

\textit{\textbf{NB:}} Okay. Could there be a Year 7 winner, a Year 8 winner, a Year 9 winner? In the house? So you're running this. It's whole school. The house within Year 7 that gets the best answer, the most correct answers, rather than it all becoming Acton's answer, because the chances are, you're gonna get a lot...because there's only five year groups, chances are you're going to get probably an even split in there somewhere. Five is better than six, but still, you're probably gonna get some answers that are all the same.

\textit{\textbf{DR:}} But if the questions were targeted in such a way that that wouldn't happen? The Henry VIII isn't a very good example, because everybody knows that.

\textit{\textbf{NB:}} Okay. So you'd more like...what's the...like guessing population or something, that you'd...is it all going to be multiple choice?

\textit{\textbf{DR:}} That would be for the people making the quiz to decide upon.

\textit{\textbf{NB:}} Okay. I think it would probably work better if you didn't give the whole of Acton an answer holistically. If you tallied it instead, so that if...7A chose four wives, and then 8A chose two wives, and then they were wrong, they don't get any points for being wrong. The ones who are right get like one point per correct answer, so then Acton...three of the year groups got it right in Acton, so Acton got three points for that question, and I think that would probably be easier to tally up if you could do it that way. So Acton got three points for that question because three of the forms got it right, and two of the forms didn't get it right, two of the year groups didn't get it right, in Acton. Could that be...?

\textit{\textbf{DR:}} That would work, absolutely. That would be a better way of handling it.

\textit{\textbf{NB:}} Okay. You alright miss?\\

\begin{center}
\textit{\textbf{*SIAN JOAH, an art teacher at the school, enters the room. SIAN and NB briefly exchange words*}}
\end{center}

\textit{\textbf{\\NB:}} And then that tally, that would just be totted up and worked out at the end.

\textit{\textbf{DR:}} That would work, yeah. I was told that the school is really aiming to emphasise the house system. So each point using your system could be converted into a house point.

\textit{\textbf{NB:}} It could, yeah. So you could tot it up, yeah. And that would probably fit in quite nicely with the value of a house credit. One form all gets one question, gets it correct, that would...and then, because there's going to be lots of questions...What you don't want to do is devalue the house credit, so you end up giving 300 house credits out for a quiz that lasted 10 minutes. But if throughout the year group, for each question, if the whole school, everybody is taking part and there's three house credits, maybe four, awarded per house, throughout the entire school, for each question, that could work out fine.

\textit{\textbf{DR:}} And how would you feel about a real time element? Say an area on the side of the screen that says ``most of Baxter have chosen this answer'', ``half of Clive have chosen this one''.

\textit{\textbf{NB:}} What, real time kind of...fluctuations in...hmm. Leading them to make different decisions?

\textit{\textbf{DR:}} Yeah. Almost tactical.

\textit{\textbf{NB:}} Yeah. Yeah, I like that. Yeah. Could you have the option of turning it off and on?

\textit{\textbf{DR:}} Yeah, I could put that in.

\textit{\textbf{NB:}} Yeah? Good. Yeah, that'd be wicked. Yeah.

\textit{\textbf{DR:}} Cool.

\textit{\textbf{NB:}} So you could have like a...little bar chart?

\textit{\textbf{DR:}} That would work, yeah.

\textit{\textbf{NB:}} \textit{\textbf{*giddy laugh*}} That'd work, that'd be wicked! Yeah. Could you have it like a...answer A, answer B, answer C, and it would be going all...woooh.

\textit{\textbf{DR:}} Yeah, that would be good.

\textit{\textbf{NB:}} 50 people are here, 5 people are here, 5 people are there, but for the whole school rather than per house maybe? So that was what everybody in the school was...or do you think it would be better to do it per house? Could you do a house one?

\textit{\textbf{DR:}} Yeah, I think so. I think doing everybody in the school on one screen would be a bit, so I...

\textit{\textbf{NB:}} So you could have Acton...imagine it on its side or something, so this is the Acton, Year 11, Year 10, Year 9, Year 8, Year 7...no, don't do that. So just Year 11, sorry Acton, and then answer A, B, C and D, and then how many people in Acton have gone for that in real time. And then for Baxter and Clive and Darwin. So you could just see these bar charts going up.

\textit{\textbf{DR:}} Or you could use dots and colours. So say each person or form in that house is a dot, then when they've picked, their dot would turn a certain colour, depending on their answer.

\textit{\textbf{NB:}} Okay. That would be a heck of a lot of dots!

\textit{\textbf{DR:}} Yeah.

\textit{\textbf{NB:}} On a graphic the size of an iPad.

\textit{\textbf{DR:}} I guess that's an issue.

\textit{\textbf{NB:}} \textit{\textbf{*laughs*}}

\textit{\textbf{DR:}} You could make them small dots.

\textit{\textbf{NB:}} Yeah! Okay, so that, like a...pull out panel?

\textit{\textbf{DR:}} Yeah, that's a good idea.

\textit{\textbf{NB:}} So the kids have got their iPad in front of them, and they've got the question, and then they can just ``phwish'', pull out the panel and then you can see what's happening as there's like 10 seconds left, you'd be like ``Oooh, uuuhh, I need to pull out the panel to...''maybe you could give them half a point if they get the answer after looking at the panel.

\textit{\textbf{DR:}} That could work.

\textit{\textbf{NB:}} Could you do that?

\textit{\textbf{DR:}} Yeah, definitely.

\textit{\textbf{NB:}} Man, that'd be wicked. So this would be like a...you know, ask the audience sort of thing, isn't it? Pull the panel out, ``right yeah, that's what I was gonna go for'', go for B because everyone else has gone for B.

\textit{\textbf{DR:}} But lose half a point for doing it.

\textit{\textbf{NB:}} Exactly. Okay. That would be great. All of that, great. But definitely try and build in some breaks, just in case.

\textit{\textbf{DR:}} Okay. And how much control would you want? Miss Blower envisioned it as all the heads of year - house, sorry - in their office, watching things on your iPad. So would you like to...

\textit{\textbf{NB:}} Have an overview?

\textit{\textbf{DR:}} Yes.

\textit{\textbf{NB:}} Definitely. How easy would it be to set up the questions?

\textit{\textbf{DR:}} That's what I was coming on to next. It would be like a...form building interface. I've not worked it all out yet - I'm still in the planning stages. But there'd be an ``add question'' button. You'd press the button, and be given an input to type the question, and then the question type. Then you'd enter the possible answers, and mark one as correct. Then you'd press the button again to add another question, or add a break, or a section, or finish the quiz.

\textit{\textbf{NB:}} Yeah. Because we've used Quizdom before, haven't we?

\textit{\textbf{DR:}} Yeah.

\textit{\textbf{NB:}} But it was...it's never really come that much out of the science department. Because everybody's got the iPads, we could roll it out in form time...it'd be really good. It's just getting everybody to have an iPad. Year 7 haven't, and next years Year 7 won't, only Year 8, Year 9 and Year 10 will have. But then, there are iPads that we can book. There's lots of iPads we can book, but not enough for the whole year group. It might be...red, green, blue, white...red, green blue, so that's 45, and we've got white, then brown, that's another 30; so we've got 75 communal iPads. Which could do almost a whole year group, one between two. So we could have Year 11s with those, Year 10 with their own, but not everybody's bought into the scheme. It's just getting the hardware out that's gonna be the problem.

\textit{\textbf{DR:}} Yeah. This wouldn't be done until September 2016, so the situation might have improved by then. And in classrooms with computers, like 16 and 17, they could just use the computers, because it will be web based.

\textit{\textbf{NB:}} Aaah.

\textit{\textbf{DR:}} But I'll package it up as an app for the iPads so it can be pushed for those who have them, and the rest can just access it via the web.

\textit{\textbf{NB:}} Yeah. Yeah, that'd be much better. 

\textit{\textbf{DR:}} How many computers are there in the school?

\textit{\textbf{NB:}} If we went with September 2016, in theory, you'd have Year 11 with their iPads, Year 10 with their iPads, Year 9 and Year 8 with their iPads. And then communal iPads, would fill in the rest of the gaps for the Year 7s, and computer rooms, that would probably be doable then.

\textit{\textbf{DR:}} Right. That's good.

\textit{\textbf{NB:}} Probably.

\textit{\textbf{DR:}} And how easy would it be to...

\textit{\textbf{NB:}} What about making it...it's cross platform, web based, so if they had a smartphone they could just do it on their smartphone?

\textit{\textbf{DR:}} Yeah.

\textit{\textbf{NB:}} Right. Perfect. Yeah, I don't see a problem with that then at all.

\textit{\textbf{DR:}} And how easy is it going to be to integrate into the current form time system?

\textit{\textbf{NB:}} Well, we've got a quiz on the form time system at the moment, and like I said we're getting the Year 11 house captains to write it. If we do a quick trial next year, of this, and we're happy with it, and the teachers are au fait with the technology, then we would just, if it were good to go, and everybody was happy with it, we'd just roll it out instead of printed pieces of paper, we'd just do that.

\textit{\textbf{DR:}} Would you and the other heads of house want to write the questions, or would that be better left to the house captains?

\textit{\textbf{NB:}} Probably the house captains. Just because they're...we need to give them something to do! \textit{\textbf{*laughs*}} No, it's probably better with the house captains, because they're going to have a better knowledge of what questions the kids are going to be more interested in. If we were gonna...I know Mr Warr is writing a program for a house day, and something like this would fit really well for that. Everybody's walking round, and they could do a set task with everybody at the same time. For the house day, this would work really well. He would write the questions for that, probably. So it could be used for both: the house day, where everybody is off timetable and doing house competitions, and form times. It could be both.

\textit{\textbf{DR:}} How simple should the whole thing be? I don't want to make it so simple that you can't do anything, but I'm worried that if it's too complicated then people won't be able to work out what to do. What's the general standard of technology members amongst the staff?

\textit{\textbf{NB:}} Well, because everybody has had to use iPads this year - every teacher knows how to use an iPad, and they're familiar with the interface, and has had to use it for teaching this year - everybody is clued up on iPads. But if the questions were already written, and the teachers were told ``right, it's quiz day today - get yourself logged in, we're starting the quiz at 5 past'', all they've got to do is make sure the kids are all on the right app. Isn't that right?

\textit{\textbf{DR:}} Yeah, that's it.