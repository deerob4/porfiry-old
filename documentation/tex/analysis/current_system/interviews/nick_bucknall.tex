\textit{\textbf{DR:}} Okay. So, for my Computing coursework, I've got to build a system, for a business. I've chosen here, because I like it here.

\textit{\textbf{NB:}} Okay. \textit{\textbf{*sniggers*}}

\textit{\textbf{DR:}} I thought a good idea for the system would be something that helps improve form times...

\textit{\textbf{NB:}} Okay...

\textit{\textbf{DR:}} And so I thought about a quiz system. I remember that last year quizzes were something that you did quite often.

\textit{\textbf{NB:}} Yep.

\textit{\textbf{DR:}} I thought that the way they were handled could have been improved.

\textit{\textbf{NB:}} Okay. Are you aware that the structure's changing?

\textit{\textbf{DR:}} Yes, with the heads of house.

\textit{\textbf{NB:}} Yes. Alright. So we were horizontal - obviously in year groups, as you know - now we're going vertical, and we're in houses. We're running a weekly quiz, that's probably going to fit in... I say weekly, but it's probably going to be three times a half term. At the moment, we're asking the house captains to make the quiz, and then they're going to dish it out, and it's supposed to be relevant for years 7 - 11. They're going to dish it out, and it's going to be done, and then marked. The problem with that is trying to make it relevant to all year groups, and, if we try and do it on the computer, not all classrooms have access to computer rooms, and...but we are obviously doing the iPads thing now. So Years 8 and 9 do have iPads, so having it on a...an app would be really useful. So what's your plan?

\textit{\textbf{DR:}} Well the idea was - and I've spoken to Mr Warr and Miss Blower about this, and they seemed to like it - part of the app would let you make the quizzes, and then the different houses would be able to find the quiz on the app, answer it, and then have the quiz marked automatically.

\textit{\textbf{NB:}} Okay, good.

\textit{\textbf{DR:}} It would then produce graphs and such. What are your initial thoughts on that?

\textit{\textbf{NB:}} Yeah, sounds alright.

\textit{\textbf{DR:}} Right. An extension of that idea, that I spoke to Miss Blower about, and she seemed really pleased with, was if there was an event, each term, where the school would get together, and each student would have an iPad, or a computer. They'd login to the system, and it would be like a live quiz. So, say the question is ``how many wives did Henry VIII have?'', that would come up on all the system's at once, and then each student would have their answer.

\textit{\textbf{NB:}} So would it be on a timer?

\textit{\textbf{DR:}} Yes.

\textit{\textbf{NB:}} What if, just working in a school environment, where things go wrong, or somebody knocks on the door and you have to speak to somebody, what would happen if there was an interruption?

\textit{\textbf{DR:}} How do you mean an interruption?

\textit{\textbf{NB:}} So if you've got all of...everybody's on their iPad, let me see if I'm understanding you right. Everybody's on their iPad, they're all in different rooms, and it automatically is generated on everybody's iPad at the same time, the question? Yep. Then if you have a certain set time, is it going to be when you've put your answer in it generates the next question, or is it going to be on a timer for the whole school?

\textit{\textbf{DR:}} Yes.

\textit{\textbf{NB:}} Okay. So my question is...if it's on a timer, you're on question 1, and somebody knocks on your door, and says ``sorry sir, so and so's just arrived in reception, can you send John down because his mum's here to pick him up'', and I'm like ``yeah okay, John go out'', and I look down and now we're on question 4, because we've missed out question 2 and 3 because I've been distracted, or the class got distracted. Could that happen?

\textit{\textbf{DR:}} Potentially, yes.

\textit{\textbf{NB:}} Okay, so you'd have to really try and minimise interruptions. While the quiz was happening you'd have to almost treat it like exam conditions.

\textit{\textbf{DR:}} Sort of. I suppose so.

\textit{\textbf{NB:}} Over the whole school.

\textit{\textbf{DR:}} I see where you're coming from.

\textit{\textbf{NB:}} I'm just trying to be awkward...

\textit{\textbf{DR:}} I'm sure.

\textit{\textbf{NB:}} ...because when we do these things, it sounds good, but because it's a school environment, things do pop up, things do happen. Somebody passes out in the classroom, and then you have to go and fix them, and then everybody's flustered, and then you look back and then when you work on a timer, it's sometimes, it's tricky to stick to. It can be. It probably would work 8 or 9 times out of ten. I mean, I've only had two kids pass out in my classroom...

\textit{\textbf{DR:}} That's pretty good!

\textit{\textbf{NB:}} ...this year.

\textit{\textbf{DR:}} Oh. Still.

\textit{\textbf{NB:}} Yeah. That's the thing - it does happen, and that would spoil something that was done on a timer. How long start to finish would you envisage the quiz lasting?

\textit{\textbf{DR:}} Well, I thought about 15 minutes. But...

\textit{\textbf{NB:}} That's quite a long time.

\textit{\textbf{DR:}} Yeah. But Miss Blower wanted half an hour, split up into rounds.

\textit{\textbf{NB:}} Right. See the rounds...the rounds thing would work if you could have a pause and a break. But then everybody would start off at the same time again?

\textit{\textbf{DR:}} Yes.

\textit{\textbf{NB:}} Okay. So you could get your kind of...anything that...somebody knocked on your door you could be like ``need two minutes''. Then you'd have a breather time. Yeah, I think breaking it up would be more manageable for a school environment, just in case something did happen.

\textit{\textbf{DR:}} So, if say the most popular answer in 8A was 4 wives, then that would become 8A's answer...\\

\begin{center}
\textit{\textbf{*NB gives out an appreciative ``ohhhh'' indicating that he has finally grasped the concept.*\\}}
\end{center}

\textit{\textbf{\\DR:}} ...and if that's then spread through the house, so say 8A, 9A and 10A chose 4 wives, then that would become Acton's answer.

\textit{\textbf{NB:}} Okay. I like that. What if they all choose something different?

\textit{\textbf{DR:}} I'd have to think about that.

\textit{\textbf{NB:}} Okay. Could there be a Year 7 winner, a Year 8 winner, a Year 9 winner? In the house? So you're running this. It's whole school. The house within Year 7 that gets the best answer, the most correct answers, rather than it all becoming Acton's answer, because the chances are, you're gonna get a lot...because there's only five year groups, chances are you're going to get probably an even split in there somewhere. Five is better than six, but still, you're probably gonna get some answers that are all the same.

\textit{\textbf{DR:}} But if the questions were targeted in such a way that that wouldn't happen? The Henry VIII isn't a very good example, because everybody knows that.

\textit{\textbf{NB:}} Okay. So you'd more like...what's the...like guessing population or something, that you'd...is it all going to be multiple choice?

\textit{\textbf{DR:}} That would be for the people making the quiz to decide upon.

\textit{\textbf{NB:}} Okay. I think it would probably work better if you didn't give the whole of Acton an answer holistically. If you tallied it instead, so that if...7A chose four wives, and then 8A chose two wives, and then they were wrong, they don't get any points for being wrong. The ones who are right get like one point per correct answer, so then Acton...three of the year groups got it right in Acton, so Acton got three points for that question, and I think that would probably be easier to tally up if you could do it that way. So Acton got three points for that question because three of the forms got it right, and two of the forms didn't get it right, two of the year groups didn't get it right, in Acton. Could that be...?

\textit{\textbf{DR:}} That would work, absolutely. That would be a better way of handling it.

\textit{\textbf{NB:}} Okay. You alright miss?\\

\begin{center}
\textit{\textbf{*SIAN JOAH, an art teacher at the school, enters the room. SIAN and NB briefly exchange words*}}
\end{center}

\textit{\textbf{\\NB:}} And then that tally, that would just be totted up and worked out at the end.