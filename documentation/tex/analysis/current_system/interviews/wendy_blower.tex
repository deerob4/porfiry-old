\textit{\textbf{DR:}} So, my project is...I've got to come up with a system for a business, in this case Priory. It's to do with form times - a sort of quiz system, an automated one. The heads of house could write a quiz, and send it out to the different house groups, for it to be answered in form. The results would be calculated, reports made, etc. That's just a basic overview - what are your thoughts?

\textit{\textbf{WB:}} It sounds like a good idea to me. If I can just clarify exactly what it is...so, during tutor time in an afternoon, I'd be told that there's gonna be a quiz, that would all be online...

\textit{\textbf{DR:}} Yes.

\textit{\textbf{WB:}} So then we'd do it as a form...

\textit{\textbf{DR:}} Yes.

\textit{\textbf{WB:}} We'd put our results into the system...

\textit{\textbf{DR:}} Yes.

\textit{\textbf{WB:}} And then somebody, one of the heads of house, could log into a system and see, well, 8A got this many, 9A got this many, or whoever...Yeah. I think that sounds like a great idea.

\textit{\textbf{DR:}} Alright, so...I'm sorry, I've got so many questions...

\textit{\textbf{WB:}} That's alright, don't worry!

\textit{\textbf{DR:}} So would you prefer the system to be the form working together as a whole, or with each individual student, perhaps with an iPad?

\textit{\textbf{WB:}} Right, I can see ads and disads to both. From an efficiency point of view, I think if it was the whole form...I think there would be less mistakes that could be made. Because then it's either the member of staff at the computer, or one student with an iPad, and then everything gets done in one go. If you then had each student with an iPad - which I think the students would prefer, because it's more fun - if a student doesn't bring their iPad with them, or it's dead, or...not everybody has one so we have to get some from IT, what happens if we can't book them? I know you could book ahead, so I think I would prefer it if it were just done centrally...although maybe if their was an option to do either, that might be good, because their might be some days where I know I can book iPads, and I know everybody's got one, so I might say ``right, today, you're all going to do it on your own iPad''. So it's a bit like...did you ever use...

\textit{\textbf{DR:}} Quizdom?

\textit{\textbf{WB:}} That's the one! I was thinking Quizlet and that didn't sound right. Yeah, Quizdom - it sounds a bit like that. Is that the kind of thing you were going for?

\textit{\textbf{DR:}} Yeah. But I probably couldn't do both.

\textit{\textbf{WB:}} Right, okay.

\textit{\textbf{DR:}} Because it is all quite complicated, and it does take a while to...

\textit{\textbf{WB:}} Is there one that's easier than another? From your point of view.

\textit{\textbf{DR:}} Yes, the one with iPads would be quite a bit harder - but it would be more rewarding.

\textit{\textbf{WB:}} Oh, God! Okay, so if I had to go with one...who is this, who is it supposed to benefit? Is it supposed to be about engaging students?

\textit{\textbf{DR:}} Yes. I was told by Mr Warr that the school is trying to focus on the house system.

\textit{\textbf{WB:}} Yeah. I think if your aim - if your main aim is about engaging students, I think you'd have to do it with the iPads. Where they each have their own iPad, feeding to a system. I think us doing it, one feeding into one thing...perhaps if the aim was more about collating points in an efficient way, maybe that. So I think it's gotta do with your aims, hasn't it? Whatever your aim is, you have to pick the one that's suited more to that. Does that make sense?

\textit{\textbf{DR:}} Yeah, absolutely. What I thought would be nice if each student gave an answer to the quiz, and then the most popular answer in that form became that form's answer, but then if that's spread across 7A, 8A, 9A and so on, the most popular answer out of those becomes that house's answer.

\textit{\textbf{WB:}} Right, I see. But then would you need all of 7A, 8A, 9A and so on - would we all have to do the quiz at the same time?

\textit{\textbf{DR:}} Yes - and that's what I was speaking to Mr Warr about: getting the right time. With the form timetable, he says there could be a slot each term, where that could be done.

\textit{\textbf{WB:}} And I think that's the thing actually - with him saying each term, I think that would work, if we knew that...I don't know, on a day in October, and a day in January, and a day in say April...if we all knew, then yeah, I think it would work. But, see, we used to have a quiz on our timetable once a week. So I suppose in my head I was thinking that we'd be doing this once a week - it just wouldn't work, because things crop up, and then somebody gets called out, whereas if everybody knows that on this particular date everybody's gotta be doing it, then I think it could work really well, and I think that would be another reason why everybody should then...hold on, I've just thought of another problem...that's why everybody should then do it on their iPads, because they'd be like ``oh yeah, we wanna beat them next door'', because everybody would be doing it. But, if we all did it at the same time, there wouldn't be enough iPads.

\textit{\textbf{DR:}} That's what I wasn't too sure about. Because, I'd heard the school has an iPad loaning scheme. How widespread is that?

\textit{\textbf{WB:}} At the moment, we've got...I think it's approximately half of Year 8 have got them, and approximately half of Year 9. As far as I'm aware, I think that will happen...so this Christmas - this is how we did it last time - this Christmas, it will then be launched to current Year 7's, because they'll be Year 8 again. So ultimately, in a couple of years time, there will be at least half a year group in every year would have an iPad. So it's whether you could do it where, if it would work like this, it might be that you say ``right, on Thursday, everybody in A, and everybody in B does it''. So 7, 8, 9, 10, 11, A; 7, 8, 9, 10, 11, B. I mean, I'd still have to work out how many iPads that is, but it's whether you could do A and B on one day, C and D, H and W. If you could do that, we'd then have enough iPads. Could you still get it to work?

\textit{\textbf{DR:}} I think I probably could do, yes.

\textit{\textbf{WB:}} Right, okay. Or even if you...could you do just all of Acton on one day? If there wasn't enough iPads?

\textit{\textbf{DR:}} See, I suppose you could do, but what I really wanted to focus on was the realtime nature...

\textit{\textbf{WB:}} Oh right. Yeah...

\textit{\textbf{DR:}} So you'd have a thing at the bottom saying ``Webb has chosen this answer''...

\textit{\textbf{WB:}} Right, I see.

\textit{\textbf{DR:}} ``Three quarters of Houseman have now answered'' - it just makes it more interesting.

\textit{\textbf{WB:}} Oh, okay. Did Mr Warr say anything about like...the number of iPads that we've got available?

\textit{\textbf{DR:}} No, I haven't spoken to him about that.

\textit{\textbf{WB:}} Right, because...is it only me that you're talking to about it, or have you got people...?

\textit{\textbf{DR:}} No, I'm speaking to Mr Warr, you, Mrs Smith, Mr Bucknall, and the IT guys [Tim Goodman].

% 9:20