% 1. Evaluation of test results
% 2. Evaluate against original objectives.
% 3. Evaluate against evaluation criteria.
% 4. Potential future improvements.
\clearpage
\part{Evaluation} % (fold)
\label{prt:evaluation_}
This section contains an extensive evaluation of different parts of the application, and the processes that went into its development. A number of areas have been evaluated, including the test results. In order to qualify the testing process as acceptable, it must be evaluated to ensure that the majority of the testing was successfull; focus will be placed on each of the individual areas of testing that were performed, namely: the unit tests for each individual component, the acceptance testing, and the full system testing. By evaluating all these areas, it will be possible to say whether the test strategy laid out was a success, and the degree to which this is true.

Also evaluated is whether the system meets the original objectives, those objectives outlined at the beginning of the project, before development began, and that took into account the research performed in the analysis section. For the system to considered a success, it must meet all or most of these objectives; this section of the evaluation attempts to determine whether or not this is the case.

The evaluation will also take into account the evaluation criteria - these criteria are additional to the original objectives, and lay out several supplementary aspects that the system should meet to be viewed in a wholly positive light; it also takes into account areas like the length of time spent on actually developing the system. This section of the evaluation therefore evaluations these critera.

Potential future improvements will also be explored. As with any system, there are aspects of it that could be improved. This section outlines several of these, taking into account whether or not they are feasable, and gives methods that could be used to implement them within a reasonable time frame.
% part evaluation_ (end)