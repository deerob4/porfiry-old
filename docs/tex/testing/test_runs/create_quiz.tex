\subsection{Create Quiz Test Runs} % (fold)
\label{sub:create_quiz_test}
This section contains the test runs performed on the quiz creator.


\subsubsection{Add Question} % (fold)
\label{ssub:add_question}
This ensures that the user is able to succesfully add a new question to the current quiz when the add question button is pressed.
\begin{figure}[!htbp]
\centering
\begin{subfigure}{0.5\textwidth}
  \centering
  \includegraphics[width=0.95\linewidth]{testing/create_quiz/add_question/before}
  \caption{Before}
  \label{fig:sub1}
\end{subfigure}%
\begin{subfigure}{0.5\textwidth}
  \centering
  \includegraphics[width=0.95\linewidth]{testing/create_quiz/add_question/after}
  \caption{After}
  \label{fig:sub2}
\end{subfigure}
\caption{Adding a question to the quiz.}
\label{fig:test}
\end{figure}
\\As the two pictures indicate, after pressing the add question button, a new question was added to the system, and this was reflectd in the label unde the question title. \textit{Success.}
% subsubsection add_question (end)

\clearpage

\subsubsection{Edit Question} % (fold)
\label{ssub:edit_question}
This ensures that the user is able to succesfully edit a question in the current quiz.
\begin{figure}[!htbp]
\begin{subfigure}{0.5\textwidth}
  \centering
  \includegraphics[width=0.95\linewidth]{testing/create_quiz/edit_question/during}
  \caption{During}
  \label{fig:sub2}
\end{subfigure}
\begin{subfigure}{0.5\textwidth}
  \centering
  \includegraphics[width=0.95\linewidth]{testing/create_quiz/edit_question/after}
  \caption{After}
  \label{fig:sub3}
\end{subfigure}
\caption{Editing a question to the quiz.}
\label{fig:test}
\end{figure}
% subsubsection edit_question (end)


\subsubsection{Delete Question} % (fold)
\label{ssub:delete_question}
This ensures that the user is able to succesfully delete their questions from the current quiz.
\\\\\textit{\textbf{Note:} In order to prove that the question has actually been deleted, the screenshot captures the developer tools used in the creation of the application; an action showing that the question has been deleted is clearly visible.}
\begin{figure}[!htbp]
\centering
\begin{subfigure}{0.5\textwidth}
  \centering
  \includegraphics[width=0.95\linewidth]{testing/create_quiz/delete_question/before}
  \caption{Before}
  \label{fig:sub1}
\end{subfigure}%
\begin{subfigure}{0.5\textwidth}
  \centering
  \includegraphics[width=0.95\linewidth]{testing/create_quiz/delete_question/after}
  \caption{After}
  \label{fig:sub2}
\end{subfigure}
\caption{Delete a question in the quiz.}
\label{fig:test}
\end{figure}
\\The second screenshot shows that a delete question action was passed to the quiz reducer, resulting in the question being successfully deleted; this is also reflected in the labels. \textit{Success.}
% subsubsection delete_question (end)


\subsubsection{Add Category} % (fold)
\label{ssub:add_category}
This ensures that the user is able to succesfully add a custom category to the current quiz.
\begin{figure}[!htbp]
\centering
\begin{subfigure}{0.5\textwidth}
  \centering
  \includegraphics[width=0.95\linewidth]{testing/create_quiz/add_category/during}
  \caption{During}
  \label{fig:sub1}
\end{subfigure}%
\begin{subfigure}{0.5\textwidth}
  \centering
  \includegraphics[width=0.95\linewidth]{testing/create_quiz/add_category/after}
  \caption{After}
  \label{fig:sub2}
\end{subfigure}
\caption{Adding a category to the quiz.}
\label{fig:test}
\end{figure}
As expected, a dialog to enter a category name is shown when the add category button is pressed, and the new category is added successfully. \textit{Success.}
% subsubsection add_category (end)


\subsubsection{Delete Category} % (fold)
\label{ssub:add_category}
This ensures that the user is able to succesfully delete their custom categories in the current quiz.
\begin{figure}[!htbp]
\centering
\begin{subfigure}{0.5\textwidth}
  \centering
  \includegraphics[width=0.95\linewidth]{testing/create_quiz/delete_category/before}
  \caption{Before}
  \label{fig:sub1}
\end{subfigure}%
\begin{subfigure}{0.5\textwidth}
  \centering
  \includegraphics[width=0.95\linewidth]{testing/create_quiz/delete_category/after}
  \caption{After}
  \label{fig:sub2}
\end{subfigure}
\caption{Deleting a category from the quiz.}
\label{fig:test}
\end{figure}
As expected, a dialog to enter a category name is shown when the add category button is pressed, and the new category is added successfully. \textit{Success.}
% subsubsection add_category (end)


\subsubsection{Rename Category} % (fold)
\label{ssub:add_category}
This ensures that the user is able to succesfully rename their custom categories in the current quiz.
\begin{figure}[!htbp]
\centering
\begin{subfigure}{0.5\textwidth}
  \centering
  \includegraphics[width=0.95\linewidth]{testing/create_quiz/rename_category/during}
  \caption{During}
  \label{fig:sub1}
\end{subfigure}%
\begin{subfigure}{0.5\textwidth}
  \centering
  \includegraphics[width=0.95\linewidth]{testing/create_quiz/rename_category/after}
  \caption{After}
  \label{fig:sub2}
\end{subfigure}
\caption{Renaming a category in the quiz.}
\label{fig:test}
\end{figure}
As expected, a dialog to enter the new category name is shown when the rename category button is pressed, and category is renamed successfully. \textit{Success.}
% subsubsection add_category (end)

\subsubsection{Edit Answer} % (fold)
\label{ssub:add_category}
This test ensures that the user is able to edit the body of an answer in the current quiz.
\begin{figure}[!htbp]
\centering
\begin{subfigure}{0.5\textwidth}
  \centering
  \includegraphics[width=0.95\linewidth]{testing/create_quiz/edit_answer/during}
  \caption{During}
  \label{fig:sub1}
\end{subfigure}%
\begin{subfigure}{0.5\textwidth}
  \centering
  \includegraphics[width=0.95\linewidth]{testing/create_quiz/edit_answer/after}
  \caption{After}
  \label{fig:sub2}
\end{subfigure}
\caption{Marking an answer in as correct.}
\label{fig:test}
\end{figure}
\\As expected, the application allowed for the answer body to be edited, and then persisted this change after leaving the edit mode. \textit{Success.}
% subsubsection add_category (end)


\subsubsection{Mark Answer as Correct} % (fold)
\label{ssub:add_category}
This ensures that the user is able to succesfully mark an answer as correct in the current quiz.
\begin{figure}[!htbp]
\centering
\begin{subfigure}{0.5\textwidth}
  \centering
  \includegraphics[width=0.95\linewidth]{testing/create_quiz/change_answer/before}
  \caption{During}
  \label{fig:sub1}
\end{subfigure}%
\begin{subfigure}{0.5\textwidth}
  \centering
  \includegraphics[width=0.95\linewidth]{testing/create_quiz/change_answer/after}
  \caption{After}
  \label{fig:sub2}
\end{subfigure}
\caption{Marking an answer in as correct.}
\label{fig:test}
\end{figure}
\\As expected, the correct mark moved from the third to the first question, meaning that it was marked as correct. \textit{Success.}
% subsubsection add_category (end)


\subsubsection{Save Quiz} % (fold)
\label{ssub:add_category}
This test ensures that the user is able to save their quizzes to the database.
\begin{figure}[!htbp]
\centering
\begin{subfigure}{0.5\textwidth}
  \centering
  \includegraphics[width=0.95\linewidth]{testing/create_quiz/save_quiz/before}
  \caption{During}
  \label{fig:sub1}
\end{subfigure}%
\begin{subfigure}{0.5\textwidth}
  \centering
  \includegraphics[width=0.95\linewidth]{testing/create_quiz/save_quiz/after}
  \caption{After}
  \label{fig:sub2}
\end{subfigure}
\caption{Saving a quiz to the database.}
\label{fig:test}
\end{figure}
\\As expected, the correct mark moved from the third to the first question, meaning that it was marked as correct. \textit{Success.}
% subsubsection add_category (end)
