% When thinking of an objective, consider three areas
% Why is the feature being implemented?
% Who is the feature for?
% What does the feature do?

\section{Objectives}
The following objectives are based on research gathered in the interviews, observations and questionnaires that were carried out on the school, and so present an accurate picture of the features the school would find most desirable. Certain additional features have been added, as they would increase the utility of the system to even greater heights. If, at the end of development, these features have not been implemented, it would be impossible to refer to the system as a real success. The system must:\\

\begin{itemize}
	\item Provide staff with an attractive user interface with which they can create, update and delete quizzes. 

		\begin{itemize}
			\item Quizzes should allow for an unlimited number of multiple choice questions to be included, each of which should contain four possible answers - one of which is correct.

			\item Staff should be able to set how long they want to give students to answer the question, before the quiz moves on.

			\item Staff should have the ability to define an unlimited number of categories for the quiz, such as ``history'' or ``sport''; questions should be able to be placed in one of these categories.

			\item Staff should be able to specify a date and time when exactly a quiz should begin.
		\end{itemize}

	\item Allow students of the school to answer the quiz in real time and compete against one another.

		\begin{itemize}
			\item The system should support up to 1000 concurrent students partaking in the quiz without failing.

			\item The quiz should begin at the specified time no matter how many users are connected. A five minute grace period should be given to take into account valid excuses for lateness.

			\item The students should be presented with a clear and attractive interface that displays the current category, the question, and the possible answers. It should show the remaining time that students have to choose an answer.

			\item When the time has elapsed for a question, the system should highlight the correct answer, and then immediately move onto the next question.

			\item Staff should have the ability to define an unlimited number of categories for the quiz, such as ``history'' or ``sport''; questions should be able to be placed in one of these categories.

			\item All of the above should take place simultaneously, on every individual screen.
			
			\item At the end of the quiz, the system should display the winning house, as well as the number of house points earned by each house.
		\end{itemize}

	\item Display a real time visualisation of how other participants in the quiz are answering.

		\begin{itemize}
			\item During the quiz, a pull out section should be available that displays the answers that other form groups in the school have chosen.

			\item Every form group in the 

			\item The students should be presented with a clear and attractive interface that displays the current category, the question, and the possible answers. It should show the remaining time that students have to choose an answer.

			\item When the time has elapsed for a question, the system should highlight the correct answer, and then immediately move onto the next question.

			\item Staff should have the ability to define an unlimited number of categories for the quiz, such as ``history'' or ``sport''; questions should be able to be placed in one of these categories.

			\item All of the above should take place simultaneously, on every individual screen.
		\end{itemize}

	\item Work effectively across a range of different devices and display types. The school plans to use the system across tablets and desktop computers. No matter which device a user is using, all aspects of the system should be available and fully functional.

	\item Theme itself to match the house colours of the logged in user. For example, the interface should be orange for those who belong to Baxter, green for Clive, blue for Acton, and so on.
\end{itemize}
