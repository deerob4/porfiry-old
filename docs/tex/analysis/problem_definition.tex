\section{Problem Definition}

\subsection{Broad Aims} % (fold)
\label{sub:broad_aims}
\begin{enumerate}
	\item Allow staff to create multiple choice quizzes for the year groups.
	\item Give each individual form group their own user account.
	\item Allow every student in the school to participate in the quiz together.
	\item Display a live visualisation showing the current state of the quiz.
	\item Calculate the winning house of the quiz overall and within each year group.
\end{enumerate}
% subsection broad_aims (end)

\subsection{Possible Limitations} % (fold)
\label{sub:possible_limitations}
There are a number of limitations that the system will undoubtedly run into, mostly due to the limited amount of time available to develop the application, or because of the complexity of a feature..

One such limitation is the inability to provide an in-depth administration view for the heads of year. Such a view would allow them to do things like update the details of individual form groups, and add new groups through a dedicated interface. Though this would undoubtedly come in useful, and would not be especially difficult to implement, it would take additional time to build and test, time which would be better spent on improving the core quiz building and answering functionality.

Another feature that the application will not include will be the ability to automatically generate a quiz, using constraints set by the head of year. Whilst such functionality would help save the head of year time - they would not have to spend time creating their own individual quiz - it would be difficult to implement, as an algorithm would have to be written that selects the most appropriate questions for the year group, and so on. It would be a more beneficial use of time to make the actual quiz creation page easier to use, effectively avoiding the problem mentioned above.

Additionally, the application will not include quiz sharing functionality. Several other systems of this type include the ability to upload one's quiz to a publicly available site, allowing others to view and use the quiz. Though, once again, this feature would be helpful - and would actually be trivial to implement - it goes somewhat beyond the scope of the application, targeted, as it is, at a single school. A more suitable feature would be showing a list of recent quizzes and questions.
% subsection possible_limitations (end)