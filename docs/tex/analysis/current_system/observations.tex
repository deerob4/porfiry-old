\subsection{Observations}

Having been a member of the school community for over five years, I am well placed to provide an observation on how the school currently goes about creating, setting and analysing quizzes used in form times. Currently, no formal system is in place; an ad-hoc system is used, following this general pattern:

\begin{enumerate}
	\item The head of year creating the quiz thinks of a set of questions and possible answers, usually following a theme, and then writes them down on a Microsoft Word document. The correct answer is marked out, to aid the form tutor in marking the quiz. This document is then saved to a drive on the school's LAN.

	\item The head of year then notifies the individual form tutors of the quiz, usually at one of their weekly meetings, and tells them to conduct the quiz with their form group on a certain date.

	\item When the date is reached, the form tutor opens the document from the network, ensuring that the document is kept hidden to avoid members of the form viewing the correct answers.

	\item The form tutor reads each question out in turn, and the members of the form work together to attempt to work out the answer. They either come up with their own answer or choose from a list of options, depending on whether or not the question is multiple choice. The form tutor marks down the answer they chose, and this process repeats until the quiz is completed.

	\item Once all the questions have been answered, the form tutor adds up the total number of marks achieved by the form.

	\item The form tutor then passes the mark onto the head of year, either by email or when passing them in the corridor or the staffroom. This task is sometimes performed by a member of the form themselves, occasionally with the expectation that the mark achieved will be exaggerated somewhat.

	\item After receiving all the results, the head of year works out which form achieved the highest result, and which the lowest. This result is reported back to the year in the weekly assembly, often with a small reward for the highest achieving form.
\end{enumerate}