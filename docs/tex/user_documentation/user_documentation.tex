\clearpage

\part{User Documentation} % (fold)
\label{prt:user_ _documentation_}
This section outlines how to install and make use of the application.

\section{Installation} % (fold)
\label{sec:installation}
This section details how to install the application. The process is rather specialised, and the precise details vary depending on which cloud provider is being used to host the system, but by and large there is a process that can be followed.

\textit{\textbf{Note:} Installation of the system is only possible on Unix based operating systems, such as a Linux distribution or OSX.}

\subsection{Cloning the Repository} % (fold)
\label{sub:cloning_the_repository}
In order to get the system on your server, the git repository needs to be cloned. This process is simple: navigate to the folder in which the system will reside (the home directory is recommended), and run the following command:

\begin{verbatim} git clone http://github.com/deerob4/porfiry \end{verbatim}

You will be asked for credentials; these will be supplied with the system.
% subsection cloning_the_repository (end)

\subsection{Installing Dependencies} % (fold)
\label{sub:installing_dependencies}
The next stage is to install the application's external dependencies. Without these, the system will not be able to run. Ensure that Node is installed on the server; if not, install it using the appropriate package manager. The minimum version of Node required is 4.2.3. To install the dependencies, run the following command:

\begin{verbatim}npm install\end{verbatim}

This will create an \textit{npm\_modules} directory in the root of the system's folder, which will contain all of the dependencies. The system will pull from this directory, so it is important that it is not moved.
% subsection installing_dependencies (end)

% section installation (end)

\section{Usage} % (fold)
\label{sec:usage}
This section displays how to use the quiz system for it's intended purpose. The different parts of the system have been split up into different subsections for convenience.

\subsection{Creating a New Quiz} % (fold)
\label{sub:creating_a_quiz}
Creating a new quiz is a very simple process. Select your house and year from the drop down lists on the menu, and then press the ``Create a new quiz'' button.
% subsection creating_a_quiz (end)

\subsection{Editing a Quiz} % (fold)
\label{sub:editing_a_quiz}
The interface for adding questions and answers to a quiz is similar to other quiz creation systems you may have used in the past. When a quiz is first created, there is one question, with four answers - all of these can be edited.

\subsubsection{Editing a Question} % (fold)
\label{ssub:editing_a_question}
To edit a question's title, simply click or tap the current title, and type in your changes.
% subsubsection editing_a_question (end)

\subsubsection{Editing an Answer} % (fold)
\label{ssub:editing_an_answer}
To edit a question's answer, simply click or tap the current answer, and type in your changes.
% subsubsection editing_an_answer (end)

\subsubsection{Marking an Answer as Correct} % (fold)
\label{ssub:marking_an_answer_as_correct}
To mark an answer as correct, tap the tick icon at the end of the answer. The answer currently marked as correct will no longer be so.
% subsubsection marking_an_answer_as_correct (end)

\subsubsection{Adding a Question} % (fold)
\label{ssub:adding_a_question}
To add a new question, simply press the ``Add Question'' button.
% subsubsection adding_a_question (end)

% subsection editing_a_quiz (end)

\subsection{Scheduling a Quiz} % (fold)
\label{sub:scheduling_a_quiz}
To schedule a quiz, perform the following steps:

\begin{enumerate}
\item Tap the ``Quiz settings'' button.
\item Enter the date and time you wish the quiz to begin.
\item Tap the ``Save settings'' button to exit the settings panel.
\item Tap the ``Save quiz'' button.
\end{enumerate}

Your quiz has now been scheduled, and will take place at the set time.
% subsection scheduling_a_quiz (end)

\subsection{Loading a Quiz} % (fold)
\label{sub:loading_a_quiz}
If you wish to load a quiz that has been created earlier, for further editing, press the ``Load quiz'' button on the main menu, and locate the quiz you wish to edit by its title and schedule date. From there, press the ``Edit quiz'' button, and you will be taken to the familiar quiz editing interface.
% subsection loading_a_quiz (end)

\subsection{Deleting a Quiz} % (fold)
\label{sub:deleting_a_quiz}
Deleting a quiz is also very simple. Press the ``Load quiz'' button on the main menu, and locate the quiz you wish to edit by its title and schedule date. From there, press the ``Delete quiz'' button, and the quiz will be deleted.

\textit{\textbf{Note:} If the quiz has been scheduled to play, deleting the quiz will cancel the schedule, and the quiz will no longer take place.}
% subsection deleting_a_quiz (end)
% section usage (end)
% part user_ _documentation_ (end)
