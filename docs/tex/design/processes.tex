\section{Process Design}
This section features pseudocode for the vast array of processes that will be used throughout the system. Due to the functional paradigm with which the application will be programmed, many of these processes (referred to hereafter as functions) will be used multiple times, and combined with other functions. To aid in readability, pseudocode for each function will be listed only once.

\subsection{API Functions}
Though a relatively small section of the application, the API nevertheless serves a vital role. Reading and writing from the database, the API routes will return the raw data for the quiz.

\subsubsection{Create Entity}
This function will save an entity to the database.
\begin{verbatim}
process post (entity, request) {
  create new instance of entity
  set properties of entity to request properties
  call database save function
  if error occurs, return error
  else return "Entity created!"
}
\end{verbatim}

\subsubsection{View Entities}
This function will return all entities of a specified type in a JSON file.
\begin{verbatim}
process get (entity) {
  call database find query for entity
  if error occurs, return error
  else return JSON
}
\end{verbatim}

\subsubsection{View Single Entity}
This function will return a single entity of a specified type in a JSON file.
\begin{verbatim}
process get (entity, id) {
  call database find query for entity using ID
  if error occurs, return error
  else return JSON
}
\end{verbatim}

\subsection{Validation}
Validation will be used on every input throughout the system. This section lists the psuedocode that will power the different validation functions.

\subsubsection{Presence Check}
This function will return true if a field contains a value.
\begin{verbatim}
process presenceCheck (field) {
  if (field.length) {
    return true
  } else {
    return "A value must be entered."
  }
}
\end{verbatim}

\subsubsection{Range Check}
This function will return true if a field's value is greater than a specified value.
\begin{verbatim}
process rangeCheck (field, max)
  if (field.body > max) {
    return true
  } else {
    return "Value must be greater than 0."
  }
}
\end{verbatim}

\subsubsection{Date Check}
This function will return true if a field's date is before the current date.
\begin{verbatim}
process validDate (field) {
  set currentDate = the current date
  if (field.date before currentDate) {
    return "Date must be after current date."
  } else {
    return true
  }
}
\end{verbatim}
