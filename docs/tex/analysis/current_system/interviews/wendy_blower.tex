\textit{\textbf{DR:}} So, my project is...I've got to come up with a system for a business, in this case Priory. It's to do with form times - a sort of quiz system, an automated one. The heads of house could write a quiz, and send it out to the different house groups, for it to be answered in form. The results would be calculated, reports made, etc. That's just a basic overview - what are your thoughts?

\textit{\textbf{WB:}} It sounds like a good idea to me. If I can just clarify exactly what it is...so, during tutor time in an afternoon, I'd be told that there's gonna be a quiz, that would all be online...

\textit{\textbf{DR:}} Yes.

\textit{\textbf{WB:}} So then we'd do it as a form...

\textit{\textbf{DR:}} Yes.

\textit{\textbf{WB:}} We'd put our results into the system...

\textit{\textbf{DR:}} Yes.

\textit{\textbf{WB:}} And then somebody, one of the heads of house, could log into a system and see, well, 8A got this many, 9A got this many, or whoever...Yeah. I think that sounds like a great idea.

\textit{\textbf{DR:}} Alright, so...I'm sorry, I've got so many questions...

\textit{\textbf{WB:}} That's alright, don't worry!

\textit{\textbf{DR:}} So would you prefer the system to be the form working together as a whole, or with each individual student, perhaps with an iPad?

\textit{\textbf{WB:}} Right, I can see ads and disads to both. From an efficiency point of view, I think if it was the whole form...I think there would be less mistakes that could be made. Because then it's either the member of staff at the computer, or one student with an iPad, and then everything gets done in one go. If you then had each student with an iPad - which I think the students would prefer, because it's more fun - if a student doesn't bring their iPad with them, or it's dead, or...not everybody has one so we have to get some from IT, what happens if we can't book them? I know you could book ahead, so I think I would prefer it if it were just done centrally...although maybe if their was an option to do either, that might be good, because their might be some days where I know I can book iPads, and I know everybody's got one, so I might say ``right, today, you're all going to do it on your own iPad''. So it's a bit like...did you ever use...

\textit{\textbf{DR:}} Quizdom?

\textit{\textbf{WB:}} That's the one! I was thinking Quizlet and that didn't sound right. Yeah, Quizdom - it sounds a bit like that. Is that the kind of thing you were going for?

\textit{\textbf{DR:}} Yeah. But I probably couldn't do both.

\textit{\textbf{WB:}} Right, okay.

\textit{\textbf{DR:}} Because it is all quite complicated, and it does take a while to...

\textit{\textbf{WB:}} Is there one that's easier than another? From your point of view.

\textit{\textbf{DR:}} Yes, the one with iPads would be quite a bit harder - but it would be more rewarding.

\textit{\textbf{WB:}} Oh, God! Okay, so if I had to go with one...who is this, who is it supposed to benefit? Is it supposed to be about engaging students?

\textit{\textbf{DR:}} Yes. I was told by Mr Warr that the school is trying to focus on the house system.

\textit{\textbf{WB:}} Yeah. I think if your aim - if your main aim is about engaging students, I think you'd have to do it with the iPads. Where they each have their own iPad, feeding to a system. I think us doing it, one feeding into one thing...perhaps if the aim was more about collating points in an efficient way, maybe that. So I think it's gotta do with your aims, hasn't it? Whatever your aim is, you have to pick the one that's suited more to that. Does that make sense?

\textit{\textbf{DR:}} Yeah, absolutely. What I thought would be nice if each student gave an answer to the quiz, and then the most popular answer in that form became that form's answer, but then if that's spread across 7A, 8A, 9A and so on, the most popular answer out of those becomes that house's answer.

\textit{\textbf{WB:}} Right, I see. But then would you need all of 7A, 8A, 9A and so on - would we all have to do the quiz at the same time?

\textit{\textbf{DR:}} Yes - and that's what I was speaking to Mr Warr about: getting the right time. With the form timetable, he says there could be a slot each term, where that could be done.

\textit{\textbf{WB:}} And I think that's the thing actually - with him saying each term, I think that would work, if we knew that...I don't know, on a day in October, and a day in January, and a day in say April...if we all knew, then yeah, I think it would work. But, see, we used to have a quiz on our timetable once a week. So I suppose in my head I was thinking that we'd be doing this once a week - it just wouldn't work, because things crop up, and then somebody gets called out, whereas if everybody knows that on this particular date everybody's gotta be doing it, then I think it could work really well, and I think that would be another reason why everybody should then...hold on, I've just thought of another problem...that's why everybody should then do it on their iPads, because they'd be like ``oh yeah, we wanna beat them next door'', because everybody would be doing it. But, if we all did it at the same time, there wouldn't be enough iPads.

\textit{\textbf{DR:}} That's what I wasn't too sure about. Because, I'd heard the school has an iPad loaning scheme. How widespread is that?

\textit{\textbf{WB:}} At the moment, we've got...I think it's approximately half of Year 8 have got them, and approximately half of Year 9. As far as I'm aware, I think that will happen...so this Christmas - this is how we did it last time - this Christmas, it will then be launched to current Year 7's, because they'll be Year 8 again. So ultimately, in a couple of years time, there will be at least half a year group in every year would have an iPad. So it's whether you could do it where, if it would work like this, it might be that you say ``right, on Thursday, everybody in A, and everybody in B does it''. So 7, 8, 9, 10, 11, A; 7, 8, 9, 10, 11, B. I mean, I'd still have to work out how many iPads that is, but it's whether you could do A and B on one day, C and D, H and W. If you could do that, we'd then have enough iPads. Could you still get it to work?

\textit{\textbf{DR:}} I think I probably could do, yes.

\textit{\textbf{WB:}} Right, okay. Or even if you...could you do just all of Acton on one day? If there wasn't enough iPads?

\textit{\textbf{DR:}} See, I suppose you could do, but what I really wanted to focus on was the realtime nature...

\textit{\textbf{WB:}} Oh right. Yeah...

\textit{\textbf{DR:}} So you'd have a thing at the bottom saying ``Webb has chosen this answer''...

\textit{\textbf{WB:}} Right, I see.

\textit{\textbf{DR:}} ``Three quarters of Houseman have now answered'' - it just makes it more interesting.

\textit{\textbf{WB:}} Oh, okay. Did Mr Warr say anything about like...the number of iPads that we've got available?

\textit{\textbf{DR:}} No, I haven't spoken to him about that.

\textit{\textbf{WB:}} Right, because...is it only me that you're talking to about it, or have you got people...?

\textit{\textbf{DR:}} No, I'm speaking to Mr Warr, you, Mrs Smith, Mr Bucknall, and the IT guys [Tim Alexander].

\textit{\textbf{WB:}} Right, okay. So in fact...it might be the IT guys might be able to help you out a little bit more there - because they know exactly how many iPads we've got. Yeah, that's more the technical side then, isn't it? But yeah, I agree actually - if you're focusing on the real time...I mean, I've been in a school before where we did...it was a talent show, and everybody scored at the same time. I can't even remember how we did it, but everybody scored at the same time, so you know like you see it on the TV? When you look on the screen, and it's like ``who's voted for person A'', and all the scores start going up, and you all vote at the same time.

\textit{\textbf{DR:}} That's kind of what I had in mind.

\textit{\textbf{WB:}} Yeah. So I see how that would be so good, and everybody would love it, to be able to see...yeah. So the IT guys are your best bet for talking about it. I mean, how long do you think this would take to make?

\textit{\textbf{DR:}} Well, the coursework officially has to be finished by the end of the Easter term. By that point, I should have a mostly working, fully tested system, but I'll need to do a bit of extra polish and work out a deployment strategy with the IT guys before it's ready for the school to use.

\textit{\textbf{WB:}} Okay. So I think in terms of real life then, if we were going to use this moving forward, we're at least a year away then, aren't we? By which time we might have more iPads. I mean, I'm looking at getting rid of all my laptops, and replacing them with iPads, and that certainly can't be done by tomorrow. So it might be we've got more iPads in school by then anyway, so it might work.

\textit{\textbf{DR:}} And bear in mind, in Mrs Smith's and Mr Edge's rooms, they can use the computers, because it would be web based.

\textit{\textbf{WB:}} Oh, so we could use the computer rooms, like 22 and T1?

\textit{\textbf{DR:}} Yes.

\textit{\textbf{WB:}} And then of course we've got laptops in school. Oh, okay. Okay, yeah. You're idea is good then, that it's the real time.

\textit{\textbf{DR:}} So, how well do you think students would take to this sort of system? I did the quizzes last year, just on paper, and I never thought it was that much fun.

\textit{\textbf{WB:}} I've got a Year 8 tutor group, and I think they would love it. Have you thought about talking to any students?

\textit{\textbf{DR:}} I emailed Mrs Smith a questionnaire to hand out, but I haven't heard back yet.

\textit{\textbf{WB:}} We have got a Year 10 student you could ask in the room...

\textit{\textbf{DR:}} That could be very helpful.\\

\begin{center}
\textit{\textbf{BETH, a Year 10 student, joins the interview.\\}}
\end{center}

\textit{\textbf{\\WB:}} Beth!

\textit{\textbf{BETH:}} Yes?

\textit{\textbf{WB:}} I'll try and talk a bit louder so you can hear me on your ear phones. Right, just picture it. You know when you've done quizzes in form time before, and it's a piece of paper, or your teachers reading out a quiz, imagine that scenario in your form room. How does everybody feel about it? Are you bored? Do you like it? Does everybody get involved?

\textit{\textbf{BETH:}} Not really.

\textit{\textbf{WB:}} Do you know the reason why?

\textit{\textbf{BETH:}} It's just quite boring. The questions aren't very good.

\textit{\textbf{WB:}} Okay. So imagine the same questions - pretend the questions are the same - but this time, you're all voting for answers via...would it be an app?

\textit{\textbf{DR:}} Yes.

\textit{\textbf{WB:}} Via an app, or you're on a laptop doing it, and the whole school's doing it at the same time. So you do it in your house - who's your house?

\textit{\textbf{BETH:}} Webb.

\textit{\textbf{WB:}} So you'd be doing it for Webb, and real time you would be able to see on the board what everybody's answered, and who's winning?

\textit{\textbf{DR:}} Yes.

\textit{\textbf{WB:}} Would that make you more...want to do it more?

\textit{\textbf{BETH:}} I think so, yeah. I've got quite a competitive form, so if we could see our results live against the other forms, that would make it a lot more interesting.

\textit{\textbf{WB:}} Right. Is that from me saying that you're competing with everybody else, or would it help that it's real time and you're using apps and stuff?

\textit{\textbf{BETH:}} I think it's more that we can see what everyone else is doing, and that it has a bit more importance. At the moment they just get thrown in the bin. But if it's for house points and actually matters, then that would be better.

\textit{\textbf{WB:}} Right. So your age group...'cause I'm just saying to Dee here that my tutor group are in Year 8, and I know they'd love to do that, where they've got apps and they can answer. I know they'd love it, but of course they're a couple of years younger than you. So you think the competitive...

\textit{\textbf{BETH:}} Yeah.

\textit{\textbf{WB:}} It doesn't matter though that you'd be in Year 11, you'd still want to do it?

\textit{\textbf{BETH:}} Yes, if it's competitive enough.

\textit{\textbf{WB:}} Okay. Thank you!\\

\begin{center}
\textit{\textbf{BETH leaves the interview.\\}}
\end{center}

\textit{\textbf{\\WB:}} Sorry, I feel like I'm taking over the interview! \textit{\textbf{*laughs*}}

\textit{\textbf{DR:}} No, it's good! \textit{\textbf{*laughs*}}

\textit{\textbf{WB:}} \textit{\textbf{*laughs*}}

\textit{\textbf{DR:}} Should the individual tutors be able to help make the questions, or should that be left down to the heads of house? Before, with the head of year system, this would have worked a bit more nicely, because there's one key figure per year, but...

\textit{\textbf{WB:}} Hmmm. Good question. I don't think form tutors should be involved. I think we could be involved in...say like Beth's just been saying, they'd have like the questions, so it might be nice if, lets say...so I'm in Acton, so it would be good if in an Acton meeting, so there's like me, Mr Warr, Miss Jebb, Miss Hayman; and a new guy, who'll be starting in September; so there's the five of us and Mrs Smyth, she's head of house...it would be good if in a meeting we could say ``right, can we make sure that we've got...a section on history, a section on reality TV'', that kind of thing, so it's not all...like some of our quizzes, like Beth said, some of them are a bit boring. I don't know where the questions come from, whether they've just been taken of the internet, but they're a bit boring. So it'd be nice if we could have an input like that, but I wouldn't wanna have an input in the questions, because I know I wouldn't cheat \textit{\textbf{*laughs*}}

\textit{\textbf{DR:}} I see \textit{\textbf{*laughs*}}

\textit{\textbf{WB:}} Right! But of course, if I've got my form and they're all getting excited, it would be very hard to not guide them in the right direction, and I think it then would be unfair. Whereas if the heads of houses had come up with the questions, they could all be in their office, watching real time what's going on, and shouting at the screen! But they've got no input, and we didn't know what the questions were either, and it means we can get involved with what's going on, rather than it being them and us, kind of thing. We could get involved with them then.

\textit{\textbf{DR:}} That could be really good.

\textit{\textbf{WB:}} Yeah, because we're part of the house, you know, we're supposed to be joining in, with things that are going on, like, going off topic slightly, but there's a house run that's going to be happening, and we've got to do it as well. So... \textit{\textbf{*laughs}} ...I know! I'm not looking forward to that! But we can't expect them to do it if we're not going to do it. So it would be nice actually if we didn't know the questions.

\textit{\textbf{DR:}} Okay. And how would you feel if there were some subject related questions? I spoke to Mr Warr, and he said he wouldn't mind there being maths questions in there, but it's slightly different with business, because there's less of an ability gap.

\textit{\textbf{WB:}} I like the idea of there being subject questions...I think actually business could be alright, because all you'd have to do is make sure that there are only questions asked that you know students have studied in Years 7, 8 and 9. But then the only issue that brings up is if, if the form tutor is not supposed to know the questions, we'd have to have set them, wouldn't we? So...ooh, I don't know on that one. Be interesting to see what anybody else says. It depends what's more important - is it more important for tutors to get involved with their house, and try and work out the answers; or is it more important for us to ask questions around subjects? Like, on one hand, there's lots of questions you could ask in a quiz, but d'ya know what? History questions come up in quizzes more often than business. So there could be a history question that comes up, that is to do with something that they've learned. So unless we all had to put forward one question per department or something, and then we'd have to promise that we wouldn't give the answer out...\textit{\textbf{*laughs*}}...'cause otherwise we'd have an unfair advantage, when the business question comes up, and I'm the business teacher. So we'd have to just say ``right, I'm not gonna say anything for that''.

\textit{\textbf{DR:}} But, the subject questions could make it a bit like a test. I don't want to be rude, but business isn't the most exciting...

\textit{\textbf{WB:}} \textit{\textbf{*gasp*}} Dee!

\textit{\textbf{DR:}} Well it isn't though!

\textit{\textbf{WB:}} How dare you! But you did so well!

\textit{\textbf{DR:}} Not...amazingly well!

\textit{\textbf{WB:}} \textit{\textbf{*laughs*}} Okay, so then I think what you could probably do is if there was a business question, I possibly then wouldn't ask a business question that was about Years 7, 8 and 9, but I could ask a question about Richard Branson. So maybe, still do something from topics, that's more real life. I'm sure...could everybody else do that? I'm sure they could. Maths: what's the size of a football pitch? So maybe.

\textit{\textbf{DR:}} How easy should this be? The students in Year 7 aren't going to be able to cope with an overly detailed interface.

\textit{\textbf{WB:}} Are you talking about how easy the questions should be, or how easy to use?

\textit{\textbf{DR:}} Use.

\textit{\textbf{WB:}} In my mind, something like this is as easy as...a question, you know I'm thinking of apps I've seen before, a question, and multiple choices? Have you thought about whether it's gonna be multiple choice?

\textit{\textbf{DR:}} Yes, the people who make the quiz can choose between either multiple choice, multiple correct answers; or a custom input, where the students type in an answer.

\textit{\textbf{WB:}} I think it depends on the type of question as to whether you input yourself. So if you do both, I think some should be multiple choice...but let's say there was a maths type question. If the answer is like...15, or 1974, you can't type it wrong, can you? The answer is what it is. Whereas if you're asking someone ``what happened on the 7th July?'', however many years ago it was, the way somebody would write ``it was the day that London was bombed'', or whatever, everybody would write it differently.

\textit{\textbf{DR:}} But for that kind of question, multiple choice would be a better fit.

\textit{\textbf{WB:}} Yeah. So if you could do both, depending on the answer whether it would be multiple choice or just an open, I think that would be good.

\textit{\textbf{DR:}} Okay. And how quick are the iPads to get started with?

\textit{\textbf{WB:}} They're really quick. Do you mean in terms of using it in a lesson, switch it on, start using it?

\textit{\textbf{DR:}} Yes.

\textit{\textbf{WB:}} They don't get switched off. So literally, when they have an iPad, and when they've got their own iPad, they're all on. That's why we prefer them really. With the computers, well, you remember using them - they're a pain in the bum. You switch them on, wait for them to connect...it's not like that with iPads. So you could literally do it straight away.

\textit{\textbf{DR:}} Alright. And how long do you think the quiz should last?

\textit{\textbf{WB:}} I think it should be in rounds. I think if you're doing it...if it would be something that we would do once a term, I think it would be nice if the head would say that it's going to last longer than an afternoon's form time, for example. I mean, I've just presumed it would be done in form time.

\textit{\textbf{DR:}} Yes, that's what I was thinking.

\textit{\textbf{WB:}} So I'm thinking that we'd extend form till, I don't know, 2:15. I think if it's made a big deal out of, I think they'd really get into it. And I think rounds really spices it up. Like, I've done quizzes before, I don't know if I ever did one with your class, but I've done quizzes before where you've got the picture round, so you'd have a picture round and it gets passed round, but then you start with Round 1, and Round 1 is whatever, and Round 2 is whatever, and then you've got the music round as well. So I think it would be nice to do it like that. So I think different rounds...obviously we don't want it too long. Four or five rounds, maybe? And if it was finished and everybody started lesson 5 by 2:15, that then gives you...'cause you'd want to give the results as well. I know it's live, but...there's still got to be a certain element of us all going ``oh''.

\textit{\textbf{DR:}} Yes, I was thinking that could be done in the term assembly.

\textit{\textbf{WB:}} Oh yeah, that's a good idea. If you showed live things going on, if you showed ``well, Webb answered this and Acton answered this'', you're not telling us the answers at that point? Would some people work out who'd won?

\textit{\textbf{DR:}} That's a good point.

\textit{\textbf{WB:}} I mean it might be, you know like they do on ``Who Wants to be a Millionaire?'', where you know when you ask the audience and everybody presses the buzzers, and it's like ``this many people answered A, this many people answered B, C, D'', whether you could do it like that. The only other thing is...I think students would want to know the answer straight away. Beth, would you wanna know the answer straight away for the quiz or would you be happy to wait until an assembly?

\textit{\textbf{BETH:}} I'd want to know straight after I'd answered.

\textit{\textbf{WB:}} Yeah, I think they would. Because they'd want to know whether they'd got it right. So that's the only thing. I think you might have to think about how you show the realtime and have them not figure out who's won. Unless you just did it random, on random questions.

\textit{\textbf{DR:}} That would be a bit tricky to do.

\textit{\textbf{WB:}} Oh, would it? Okay. See that's where I don't know the technical side. \textit{\textbf{*laughs*}}

\textit{\textbf{DR:}} How would you feel about there being a timer, to ensure that all forms answer within, say, 10 seconds?

\textit{\textbf{WB:}} I think that would be a good idea, because it keeps everybody on track, it means everybody's doing it at the same time, and then it just goes off to the next question if they haven't answered by then. But I think what you might have to do is do a tester. Perhaps if you did different rounds, maybe Round 1, is a test round, to show them how to do it, and show them what 10 seconds is. Because I think what might happen is, especially the very first time you do it, I think everybody would be sat ready, and it's all new, and they might go ``what, what's that question, what'' and then they're looking at you and somebody's not concentrating because their iPad's gone off. So I think you'd have to gee them up bit and give them an example and say ``right, we're starting for real now'' and then everyone will all ready to do it. But I think that's a good idea, otherwise you could have different people at different stages.

\textit{\textbf{DR:}} Right. So the quiz shouldn't start until, say, 3/4 of the forms have joined?

\textit{\textbf{WB:}} Oh, yeah, so of course you can get everybody to press a button or whatever to say that we're ready to start, like Quizdom?

\textit{\textbf{DR:}} Yes. Well, it wouldn't even need to be a button - it would just detect that they've logged in.

\textit{\textbf{WB:}} Yeah, I mean, if it was made a big deal, so like, you haven't experienced it, but we do whole school assemblies now, so every single person in the school goes to the sports hall, and we have an assembly. And it got timed one week, and the whole school got there and was sat down in 7 minutes.

\textit{\textbf{DR:}} That's pretty good.

\textit{\textbf{WB:}} Exactly. So I think if this was promoted correctly, and a couple of weeks before we were talking about it, and we were reminding our forms about it, I think we could make sure that everybody was sorted and you should get more or less 100\% of people being logged in. But I see what you're saying, I mean...no, that's not gonna be an IT thing is it? Because what you don't want to do is see that there's still one form that hasn't done it yet. But if maybe the heads of house could be sat at a station, and they could see who's signed up, if they could see that 8A hadn't signed up yet they could get on the phone to me or somebody could run down and say ``hurry up!''. You could perhaps do it like that. That might work so at least everybody's taking part, and everybody knows they've got to take part. I mean, house things, as you've probably aware from Mr Warr, it's a big deal now. So everybody will be expected to be doing stuff, and I don't think the head would be happy if he thought that the quiz could start, and there's still five forms that haven't set up yet. So I think you've got to go where everybody's on, and then you start.

\textit{\textbf{DR:}} And how likely is it that this sort of system would actually be used?

\textit{\textbf{WB:}} I think it's very likely. As long as it's easy and there's no faff. If it's a case of ``get your iPads out, open up the app''...we can get the IT guys to push a load of apps...if we said ``right, get your apps open, we want this one open'', and if it's literally and they don't...would they have to login?

\textit{\textbf{DR:}} Yes.

\textit{\textbf{WB:}} Right, that could be an issue with setting up in the first place. So we'd need time to set up in the first place...if they could perhaps do it with their school login?

\textit{\textbf{DR:}} Yeah, that's what I was thinking.

\textit{\textbf{WB:}} If they guys could sort that out, so yeah, if they could just login with their school login, then there wouldn't be an issue; there'd just be the technical things which I'll leave you and the IT guys to talk about, but I think that as long as you could just open the app, login, let's get ready to go, I don't see why people wouldn't want to use it. Because, like Beth said, it'll better than a piece of paper. Piece of paper's a bit boring - it's 2015 now. \textit{\textbf{*laughs*}} 

\textit{\textbf{DR:}} And how hard would it be to get this through management? I spoke to Mr Warr, and he's on SLT next year, so...

\textit{\textbf{WB:}} Yeah.

\textit{\textbf{DR:}} And he liked it, and I'm speaking to Mr Bucknall too, but...are they senior enough?

\textit{\textbf{WB:}} Mr Warr, as he says, he's on SLT next year, and...he's said he'll take it to SLT, has he?

\textit{\textbf{DR:}} No, but he seemed really interested.

\textit{\textbf{WB:}} Right, yeah. If he's interested, he will take it to SLT. And I think it might be worth...have you arranged a meeting with Mr Barratt?

\textit{\textbf{DR:}} No. I'm scared of him...

\textit{\textbf{WB:}} \textit{\textbf{*laughs*}} Don't need to be - he's really nice. And...I, from different conversations I've been involved with in him - see, I've been sitting on SLT this year, so I've got to know him quite well, and I think he would be really interested that you're an ex-student and you've come back, and you want to develop something for us, for one; number two, the house system. He loves the house system - we're doing the house system because of him. So I think you've got a double whammy there - it's house, plus ex-student. So I think it might be worth...if you can pluck up the courage, having a chat with him about it. Once you've done everything that you need to do, and you can say ``well I've spoken to this person, spoken to this person'', he'll see that you've used your initiative to find everything out. I couldn't imagine him saying no. Because actually, what have school gotta do? Have we actually gotta do anything? You've spoken to us, so we've given up some time, which is not a big deal, then you've got the IT guys who'll have to do the setting up of the iPads...

\textit{\textbf{DR:}} And that wouldn't be too hard.

\textit{\textbf{WB:}} Yeah. And is there anything else? Apart from the timing?

\textit{\textbf{DR:}} It's timing really.

\textit{\textbf{WB:}} Yeah. Apart from the timing, but I don't think that'll be a big issue. You know, we're doing a house run from 1:00 - 3:00 in an afternoon, so we're gonna have to change the timing of the school day...I can't see it being an issue to be honest - so I'd go for it.

\textit{\textbf{DR:}} Alright. I think that's pretty much it.

\textit{\textbf{WB:}} Okay.

\textit{\textbf{DR:}} Thank you very much for your time.

\textit{\textbf{WB:}} No problem Dee!

The interview with Blower proved helpful in understanding how form tutors and their students would react to the system. She provided helpful advice about the extent to which teachers would want to be involved, and, via Beth, provided invaluble access to a real student. Additionally, she helped to clarify whether or not the system would work more effectively via an iPad or other tablet, guiding the direction of the project. She also helped in choosing whether or not teachers should have free reign over which questions should be added, or if this should be left down to house captains.
