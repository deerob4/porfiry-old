\section{Evaluation of Evaluation Criteria} % (fold)
\label{sec:evaluation_of_evaluation_criteria}
\textit{Performance of the system should be considered. The system should be able to cope with at least 1000 connections at any one time, all communicating with each other.} The system easily met this criteria, at least on the server side. Due to the impracticality of testing the application on 1000 individual iPads, a small additional program was written that spawned socket nodes, each one simulating an iPad / other device. With the simulator generating answer packets at a rate of roughly 300/s, roughly what could be expected from a live quiz, the system was easily able to cope with the stress, generating no lag or latency in working out the updated scores.

\textit{Staff and students should find the user interface simple and easy to use.} A deal of effort was put into making the system as user friendly as possible. A consistent font, Dosis, is used throughout the entire application, and this fits well with its general light, colourful nature. Bright colours are used all the way throughout, making it stand in contrast to the usually dull and drab systems that fill the market. Additionally, subtle animations are scattered throughout, making the system more approachable to those not used to it. To further aid in this, the default quiz first displayed in the creator section acts as a tutorial, instructing users how to use the system. The actual interface was also designed with simplicity in mind: it consists mainly of a series of buttons, with labels like ``Add a question'' that make their action very obvious.

\textit{The application code should be as high quality as possible, making use of functional paradigms where possible.} A great deal of effort was spent making the code achieve this goal. One of the key tenets that has been followed is immutable state. In a program or language that makes use of global mutable state, the code can very quickly become unmanageable. Functions from any part of the application can mutate the state, making it difficult to keep track of current values at any one time. This becomes an especially big issue when using a Virtual DOM on the client side (needed to handle the large number of updates resulting from the number of connected sockets), as unnecessary renders may be invoked, causing large performance issues. Instead, a single immutable state tree has been used, from which ``smart'' components can read from, and then pass down data to purely functional components. Many of these definitions are by definition pure: given the same parameters (passed by-value as properties from parent components), they will always return the same result. The state tree can only be updated by dispatching ``actions'' - pure packets of data that describe what has happened. For example, when a new question is generated, the following structure is passed to the state reducer: \begin{verbatim} { type: 'ADD_QUESTION', body: action.body } \end{verbatim}. The reducer then returns a new copy of itself, with the new question appended to an array; in this way, immutability is achieved.

\textit{Stability is an important factor, and it will also be considered. If the system is constantly prone to crashing, it could not be called stable} Again, the system meets this criteria. Extensive testing has been performed, ensuring that all edge cases that might make the system crash have been accounted for. As noted above, the application is able to cope with a high degree of traffic without crashing. The system has been writted in such a way that it is fully automated. It will simply take the quiz passed to it, schedule jobs corresponding to the answer times, and then let them run through. In this way, maintenance is also lowered: the system can just be left running, and it will not run into problems. The environment in which it is intended, the school, is very controlled, making it even less likely that it will meet with an exception it does not know how to handle.

\textit{The cost of the system to run should also be taken into account} Though it is difficult to give a precise cost for the running of the system, due to fluctuations in hosting costs, effort has been made to make it as performant as possible. All of the stress testing for the backend was performed on a comparatively weak machine, compared to even the cheapest servers, and performance was more than acceptable. As such, the school need only rent the smallest machine, which, as was stated in the interview section, will cost around £0.33 per annum; taking this into account, this criteria is an easy success.

Overall, it would be correct to say that the evaluation criteria were met, and this part of the system was another success.
% section evaluation_of_evaluation_criteria (end)
% Fukky!