\section{Data Structures}
Due to the ease with which a quiz fits a nested structure, an object based, or NoSQL database, will be used. This type of database differs from a standard relational database in that it does not make use of tables; instead, all information pertaining to an entity is stored in a single object. This lack of traditional relationships causes issues when attempting to create things like ERD diagrams. Whilst NoSQL databases are traditionally schemaless, modelling tools will be used in this project in order to better suit the course specification.

\subsection{Quiz Structure}
Below is the structure that the actual quiz will take. The following displays the schema for the quiz, including the data types for each element. The methods of access taught in this specification are not applicable to this structure.

\subsubsection{Quiz}
This is the data dictionary / schema for the main quiz. It will be the base level of the quiz object, containing the meta-data, and a link to the categories schema.\\

\begin{table}[!htbp]
\centering
\begin{tabular}{|l|l|l|l|}
\hline
\multicolumn{1}{|c|}{{\bf Field Name}} & \multicolumn{1}{c|}{{\bf Data Type}} & \multicolumn{1}{c|}{{\bf Length}} & \multicolumn{1}{c|}{{\bf Validation}} \\ \hline
title                                  & string                               & variable                          & Presence check                        \\ \hline
startTime                              & datetime                             & variable                          & After current date                    \\ \hline
questionLength                         & integer                              & variable                          & Greater than 0                        \\ \hline
intervalLength                         & integer                              & variable                          & Greater than 0                        \\ \hline
categories                             & Array<schema>                               & variable                          & Presence check                        \\ \hline
\end{tabular}
\caption{Quiz Schema}
\end{table}

\subsubsection{Categories}
This is the data dictionary / schema for the categories section of the quiz. It will be the first level of the quiz object, containing the category name and a link to the questions schema.\\

\begin{table}[!htbp]
\centering
\begin{tabular}{|l|l|l|l|}
\hline
\multicolumn{1}{|c|}{{\bf Field Name}} & \multicolumn{1}{c|}{{\bf Data Type}} & \multicolumn{1}{c|}{{\bf Length}} & \multicolumn{1}{c|}{{\bf Validation}} \\ \hline
title & string & variable & Presence check \\ \hline
questions & Array<schema> & variable & Presence check \\ \hline
\end{tabular}
\caption{Categories Schema}
\end{table}

\subsubsection{Questions}
This is the data dictionary / schema for the questions. It will be the second level of the quiz object, containing the individual questions and a link to the answers schema.\\

\begin{table}[!htbp]
\centering
\begin{tabular}{|l|l|l|l|}
\hline
\multicolumn{1}{|c|}{{\bf Field Name}} & \multicolumn{1}{c|}{{\bf Data Type}} & \multicolumn{1}{c|}{{\bf Length}} & \multicolumn{1}{c|}{{\bf Validation}} \\ \hline
title & string & variable & Presence check \\ \hline
answers & Array<schema> & variable & Presence check \\ \hline
\end{tabular}
\caption{Questions Schema}
\end{table}

\subsubsection{Answers}
This is the data dictionary / schema for the questions. It will be the third and final level of the quiz object, containing the individual answers to a question.\\

\begin{table}[!htbp]
\centering
\begin{tabular}{|l|l|l|l|}
\hline
\multicolumn{1}{|c|}{{\bf Field Name}} & \multicolumn{1}{c|}{{\bf Data Type}} & \multicolumn{1}{c|}{{\bf Length}} & \multicolumn{1}{c|}{{\bf Validation}} \\ \hline
text & string & variable & Presence check \\ \hline
correct & Boolean & 1 & Presence check \\ \hline
\end{tabular}
\caption{Answers Schema}
\end{table}


\subsubsection{Completed Quiz Object}
By combining the above schemas, one ends up with a whole quiz, containing the different areas of the object combined into one. Taking just the field names without any data, the structure will look like this:

\begin{Verbatim}[fontsize=\small]
{
  title: String,
  start: Data,
  questionIntervals: Integer,
  intervalLength: Integer,
  categories: [
    {
      title: String,
      questions: [
        {
          title: String,
          answers: [
            { text: String, correct: Boolean },
            { text: String, correct: Boolean },
            { text: String, correct: Boolean },
            { text: String Schema, correct: Boolean }
          ]
        }
      ]
    }
  ]
}
\end{Verbatim}

As can be seen, at the beginning of the structure lies some metadata pertaining to the quiz, such as when it is due to start and the title. This is then followed by a list of categories, each of which is named, each of which itself contains the questions and answers. This nested structure fits the idea of a quiz rather nicely. The categories array holds a series of objects, each of which have a \textit{title} attribute. Next to this title lies an array of questions - the nesting indicates that a question belongs to a category - each of is another object, themselves containing an array of answers.

\subsubsection{Example Quiz}
An example quiz has been included below in order to give the reader an impression of how a real quiz might look in the database:

\begin{Verbatim}[fontsize=\small]
{
  title: 'Summer Term House Quiz',
  start: new Date(),
  questionIntervals: 10000,
  realtimeGraphics: true,
  intervalLength: 300000,
  categories: [
    {
      name: 'History',
      questions: [
        {
          title: 'Who killed JFK?',
          answers: [
            { text: 'Bill Osmond', correct: false },
            { text: 'Lee Harvey Oswald', correct: true },
            { text: 'Henry Kissinger', correct: false }
          ]
        },
        {
          title: 'Who was the British Prime Minister in 1916?',
          answers: [
            { text: 'David Lloyd George', correct: true },
            { text: 'Lee Harvey Oswald', correct: false },
            { text: 'Lord Liverpool', correct: false }
          ]
        }
      ]
    },
    {
      name: 'Sport',
      questions: [
        {
          title: 'Who won the 1966 world cup?',
          answers: [
            { text: 'England', correct: true },
            { text: 'Wales', correct: false },
            { text: 'Scotland', correct: false },
            { text: 'Uzbekistan', correct: false }
          ]
        },
        {
          title: 'Which city hosted the 1924 Summer Olympics?',
          answers: [
            { text: 'Brussels', correct: false },
            { text: 'Orelans', correct: false },
            { text: 'Madrid', correct: false },
            { text: 'Paris', correct: true }
          ]
        }
      ]
    },
    {
      name: 'Literature',
      questions: [
        {
          title: 'Who serves as the main character in "Crime and Punishment"?',
          answers: [
            { text: 'Raskolnikov', correct: true },
            { text: 'Razumikhin', correct: false },
            { text: 'Porifry Petrovich', correct: false },
            { text: 'Svidrigailov', correct: false }
          ]
        },
        {
          title: 'Which literary genre does "Ulysses" belong to?',
          answers: [
            { text: 'Modernist', correct: true },
            { text: 'Post-modernist', correct: false },
            { text: 'Romance', correct: false },
            { text: 'Thriller', correct: false }
          ]
        }
      ]
    }
  ]
}
\end{Verbatim}

\subsection{Answer Packet}
The following is the structure of the data packet that will be emitted when an answer is chosen. Due to the number of packets that will be sent in any one quiz - 22,5000 on a 25 question quiz containing 900 students - this structure is very small. It will serve as an input to practically all the algorithms throughout the system.

\begin{Verbatim}[fontsize=\small]
{
  house: Character,
  year: Integer,
  questionID: Integer,
  answer: Character
}
\end{Verbatim}

By splitting the \textit{house} and \textit{year} items, normalisation is achieved. This approach means that the form and house can be treated seperately - but computed as one value when needed - allowing for more complicated statistics to be gathered.
