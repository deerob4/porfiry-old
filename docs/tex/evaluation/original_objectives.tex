\section{Evaluation of Original Objectives} % (fold)
\label{sec:evaluation_of_original_objectives}
\textit{Provide staff with an attractive user interface with which they can create, update and delete quizzes.} In this respect, the system can be considered a success. Provision has been made for unlimited to questions to be created; the user can press the \textit{Add Question} button as many times as they like. As specified, each of these questions has four answers, only one of which can be set as correct. The ability to define categories was also implemented, allowing the quesions to be split up into natural groups. The settings panel allows for the staff producing the quiz to define a common question length, thereby setting the duration in which students can answer the question before the quiz moves on. This same panel also allows a date and time to be set, at which point the quiz will be scheduled. This schedule works, as shown in the tests.

\textit{Allow students of the school to answer the quiz in real time and compete against one another.} By and large, the objectives in this section were met. Using a simulator that spawns \textit{n} number of connections (effectively students), it was possible to check whether the quiz supports up to 1000 students at the same time without failing; the simulator proved that this is the case, and, indeed, that the system is able to support ten times that amount with no failings. Another objective was that the system should begin at the specified time, no matter how many students are connected. Due to the way the two major parts of the system, the server and the client, were written (they are decoupled, neither relying on the other), this objective was met: the system will simply move forward with each question, paying no regard to who is connected. It was decided that, rather than have a five minute grace period to allow those who were late to still take part in the quiz, the system would instead allow students to join at any time, and would simply take them to the same point in the quiz as everyone else. This promotes a ``hive'' mentality, whereby the importance of each individual node (student) is reduced; this reduces the chance of the final scores being innaccurate if the quiz is disrupted in any way. As implied above, the system was designed so that every connected client should always be at the same point in the quiz: in the same category, on the same question, with the same time remaining. This ``on-rails'' approach means that the integrity of the scores can be guaranteed, and that teachers need simply sit back and watch as the system takes over. Again, this aspect of the system was a success, the only small issue being that the timer bar jumps around a little at first when joining a quiz mid-way through a question. The interface for the quiz also meets the objectives: it contains all the required elements, including the current category name, question body, and the four answers; as well as a live timer bar that displays how much time is left to answer the question. The system also reports the final results at the end of the quiz, displaying the winning house and the points earned by each house, in the form of a bar chart.

\textit{Display a real time visualisation of how other participants in the quiz are answering.} Due to time constraints and technical difficulties, this was the only aspect of the system that was not implemented. The difficulty lay mostly in making the process efficient enough for real-world use: in order for the visualisation to work, every time someone chose an answer, every single connectec client would have to be notified of this. The real-time infrastructure makes use of the websocket protocol, so the issue was not in the size of the data being sent and received; rather, its quantity. In a quiz containing ten questions, each 10 seconds long, and with 1000 participants, 10,000 packets of data would be received by every client, with 1000 spread across ten seconds; and, more likely, half of these packets would be received across a two second period, as the majority of students choose an answer. For every one of these packets, the visualisation would have to be updated; therefore, within a 1-2 second period, the client would be updating 500 times in a single second. This is obviously an issue, and would result in so much lag that the system would be unusable. Time constraints unfortunately prevented alternative methods being explored.

\textit{Work effectively across a range of different devices and display types.} The system was designed and developed for the web, so this objective was easy to meet. The application will run effectively on all modern browsers (IE >= 10, all up to date WebKit browsers), with no missing functionality. The server aspect of the system, controlling the real-time infrastructure, can be deployed to any system for which Node binaries are available; effectively any system one could care to use. The client system need not be overly powerful; the quiz ran smoothly even on a first generation iPad Mini, with only 512mb of RAM. The school will use the system mainly on iPads, so a resolution of 1024x768 produces the best results. However, effort has been made to optimise the system for different screen sizes, and it will work with full functionality even on much smaller or larger screen sizes.

\textit{Theme itself to match the house colours of the logged in user.} This was again a success. As can be seen in the myriad screenshots of the application, the colours change correctly, according to the house of the user.

Speaking holistically, it would be remiss to state that the system did not meet the original objectives. Though it was disappointing that the live visualisations were not completed in time, practically every other aspect of the system was, and to a very high quality.
% section evaluation_of_original_objectives (end)