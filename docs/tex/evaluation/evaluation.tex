% 1. Evaluation of test results
% 2. Evaluate against original objectives.
% 3. Evaluate against evaluation criteria.
% 4. Potential future improvements.
\clearpage
\part{Evaluation} % (fold)
\label{prt:evaluation_}
This section contains an extensive evaluation of different parts of the application, and the processes that went into its development. A number of areas have been evaluated, including the test results. In order to qualify the testing process as acceptable, it must be evaluated to ensure that the majority of the testing was successful; focus will be placed on each of the individual areas of testing that were performed, namely: the unit tests for each individual component, the acceptance testing, and the full system testing. By evaluating all these areas, it will be possible to say whether the test strategy laid out was a success, and the degree to which this is true.

Also evaluated is whether the system meets the original objectives, those objectives outlined at the beginning of the project, before development began, and that took into account the research performed in the analysis section. For the system to considered a success, it must meet all or most of these objectives; this section of the evaluation attempts to determine whether or not this is the case.

The evaluation will also take into account the evaluation criteria - these criteria are additional to the original objectives, and lay out several supplementary aspects that the system should meet to be viewed in a wholly positive light; it also takes into account areas like the length of time spent on actually developing the system. This section of the evaluation therefore evaluations these critera.

Potential future improvements will also be explored. As with any system, there are aspects of it that could be improved. This section outlines several of these, taking into account whether or not they are feasable, and gives methods that could be used to implement them within a reasonable time frame.
% part evaluation_ (end)

\section{Evaluation of Original Objectives} % (fold)
\label{sec:evaluation_of_original_objectives}
\textit{Provide staff with an attractive user interface with which they can create, update and delete quizzes.} In this respect, the system can be considered a success. Provision has been made for unlimited to questions to be created; the user can press the \textit{Add Question} button as many times as they like. As specified, each of these questions has four answers, only one of which can be set as correct. The ability to define categories was also implemented, allowing the quesions to be split up into natural groups. The settings panel allows for the staff producing the quiz to define a common question length, thereby setting the duration in which students can answer the question before the quiz moves on. This same panel also allows a date and time to be set, at which point the quiz will be scheduled. This schedule works, as shown in the tests.

\textit{Allow students of the school to answer the quiz in real time and compete against one another.} By and large, the objectives in this section were met. Using a simulator that spawns \textit{n} number of connections (effectively students), it was possible to check whether the quiz supports up to 1000 students at the same time without failing; the simulator proved that this is the case, and, indeed, that the system is able to support ten times that amount with no failings. Another objective was that the system should begin at the specified time, no matter how many students are connected. Due to the way the two major parts of the system, the server and the client, were written (they are decoupled, neither relying on the other), this objective was met: the system will simply move forward with each question, paying no regard to who is connected. It was decided that, rather than have a five minute grace period to allow those who were late to still take part in the quiz, the system would instead allow students to join at any time, and would simply take them to the same point in the quiz as everyone else. This promotes a ``hive'' mentality, whereby the importance of each individual node (student) is reduced; this reduces the chance of the final scores being innaccurate if the quiz is disrupted in any way. As implied above, the system was designed so that every connected client should always be at the same point in the quiz: in the same category, on the same question, with the same time remaining. This ``on-rails'' approach means that the integrity of the scores can be guaranteed, and that teachers need simply sit back and watch as the system takes over. Again, this aspect of the system was a success, the only small issue being that the timer bar jumps around a little at first when joining a quiz mid-way through a question. The interface for the quiz also meets the objectives: it contains all the required elements, including the current category name, question body, and the four answers; as well as a live timer bar that displays how much time is left to answer the question. The system also reports the final results at the end of the quiz, displaying the winning house and the points earned by each house, in the form of a bar chart.

\textit{Display a real time visualisation of how other participants in the quiz are answering.} Due to time constraints and technical difficulties, this was the only aspect of the system that was not implemented. The difficulty lay mostly in making the process efficient enough for real-world use: in order for the visualisation to work, every time someone chose an answer, every single connectec client would have to be notified of this. The real-time infrastructure makes use of the websocket protocol, so the issue was not in the size of the data being sent and received; rather, its quantity. In a quiz containing ten questions, each 10 seconds long, and with 1000 participants, 10,000 packets of data would be received by every client, with 1000 spread across ten seconds; and, more likely, half of these packets would be received across a two second period, as the majority of students choose an answer. For every one of these packets, the visualisation would have to be updated; therefore, within a 1-2 second period, the client would be updating 500 times in a single second. This is obviously an issue, and would result in so much lag that the system would be unusable. Time constraints unfortunately prevented alternative methods being explored.

\textit{Work effectively across a range of different devices and display types.} The system was designed and developed for the web, so this objective was easy to meet. The application will run effectively on all modern browsers (IE >= 10, all up to date WebKit browsers), with no missing functionality. The server aspect of the system, controlling the real-time infrastructure, can be deployed to any system for which Node binaries are available; effectively any system one could care to use. The client system need not be overly powerful; the quiz ran smoothly even on a first generation iPad Mini, with only 512mb of RAM. The school will use the system mainly on iPads, so a resolution of 1024x768 produces the best results. However, effort has been made to optimise the system for different screen sizes, and it will work with full functionality even on much smaller or larger screen sizes.

\textit{Theme itself to match the house colours of the logged in user.} This was again a success. As can be seen in the myriad screenshots of the application, the colours change correctly, according to the house of the user.

Speaking holistically, it would be remiss to state that the system did not meet the original objectives. Though it was disappointing that the live visualisations were not completed in time, practically every other aspect of the system was, and to a very high quality.
% section evaluation_of_original_objectives (end)
\section{Evaluation Critera}
When evaluating the system, several factors will to be taken into account to determine if the system itself, and the development process, is a success. They are listed below:

\begin{itemize}
\item Performance of the system should be considered. The system should be able to cope with at least 1000 connections at any one time, all communicating with each other. This communication should all be real time and seamless, and there should not be any lags or stutters caused directly by the system.

\item Staff and students should find the user interface simple and easy to use; they should instinctively know how to use the system, with a minimum amount of training required. This will be measured by observing users as they attempt to use the different sections of the system.

\item As the school requires the system relatively quickly, it is important that development of the system is completed within a reasonable period of time. To achieve this, it is important that time is spent only on features that the school have asked for, and not on ``useful'' extras.

\item The application code should be as high quality as possible - where possible, functional paradigms, such as using \textit{map} as opposed to a loop, should be followed, and code should be kept as modular and reusable as possible, to bring in all the advantages that such a development model provides.

\item The cost of the system to run should also be taken into account. Though the school will not be charged directly for the system, there will be costs relating to its continued maintenance and uptime. If these are overly expensive, there will be a negative impact on the success of the project.

\item Stability is an important factor, and it will also be considered. If the system is constantly prone to crashing, it could not be called stable; likewise, it would also indicate a failure of the test plan.

\item The impact of the system will also be considered. If the school decides that there are no benefits to using the system - in other words, it has failed to solve the problems laid out in the analysis - the system cannot be considered a success.

\item The suitability of the test plan will also be evaluated. A test plan is important because it provides a way of ensuring the parts of the system function correctly.
\end{itemize}

If, after the completion of the project, these criteria are not met, it would be impossible, or at least very difficult, to label the system a complete success.

\section{Future Improvements} % (fold)
\label{sec:future_improvements}
% Add commentarties to test results

There are a tremendous number of ways in which the system can be improved. Perhaps the most obvious is fully implementing the live visualisation system, whereby the current status of each house and form is relayed in real time to each individual user. As set out in the evaluation above, there are a number of factors that make this relatively difficult to achieve, but it is definitely possible.

One possible feature could be the ability to see if a form or house has consistently won over several quizzes. This way, a running tally could be kept of the best form or house, increasing competitiveness between houses, one of the system's inital goals. This could be taken further with a streak system: for every \textit{n} number of quizzes the house has won, they could receive \textit{n} extra house points. Such as system would also allow individual forms that have done especially well to be reported. Of course, such an approach could make it difficult for other houses to catch up and beat the house with the highest streak: by winning just a few quizzes at first, their score would eventually snowball, placing them at an advantage during subsequent quizzes.

To solve this, a ``bonus'' system could be implemented. Every so often, special events, such as a knockout question between just two individual students, could be held, worth a large number of house points. These could take place at random intervals, perhaps between questions or categories, and would likely help to increase engagement with the quiz, an issue that was raised in the interviews. By giving out a large number of points every so often, the ``incumbent'' house with a large streak could be unfooted, making the system fairer.

A further change that might be considered is rewriting the application's structure and backend. The system currently utilises a fairly loose structure, with little coordination between files and directories. It would be beneficial to reimplement the backend in a language that enforces such a structure. A potential candidate for this might be Erlang, a language well suited for building highly concurrent, real-time applications, much like the quiz system. Roughly 40\% of the world's telephone and SMS communications pass through Erlang powererd servers, giving it a well deserved reputation of stability. The langauge also more rigidily enforces a functional style of programming. Though a, largely successful, drive to stick to the paradigm was a key criteria in the current version of the system, there are still certain areas where the code more resembles that seen in an imperative or object-oriented system, or where the code could simply be refactored to be much simpler. Using a more functional language would help improve this.

Other useful features could also be included, such as a backup feature to ensure that quizzes are never lost. Currently they are stored in only the one database, with no way to export them. Since they can easily be represented as simple JSON, it would be a simple case to include an export button, allowing the school to save their most popular quizzes to their own hard drive or storage area, ensuring that they are never lost. This would also make it easier for the school to reuse certain aspects of quizzes

This could be further improved by including a ``template'' feature, allowing quizzes to be made from other quizzes, allowing popular quizzes to be extended upon without having to start from scratch each time.

It could also be an idea to delete user data every five years automatically, ensuring that data only lasts as long as students are in the school. Otherwise, any streaks that are gained through the above improvement could be the result of ex-students from several years ago, and not an accurate reflection of the current student body.
% section future_improvements (end)

