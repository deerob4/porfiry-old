\section{Future Improvements} % (fold)
\label{sec:future_improvements}
There are a tremendous number of ways in which the system can be improved. Perhaps the most obvious is fully implementing the live visualisation system, whereby the current status of each house and form is relayed in real time to each individual user. As set out in the evaluation above, there are a number of factors that make this relatively difficult to achieve, but it is definitely possible.

A further change that might be considered is rewriting the application's structure and backend. The system currently utilises a fairly loose structure, with little coordination between files and directories. It would be beneficial to reimplement the backend in a language that enforces such a structure. A potential candidate for this might be Erlang, a language well suited for building highly concurrent, real-time applications, much like the quiz system. Roughly 40\% of the world's telephone and SMS communications pass through Erlang powererd servers, giving it a well deserved reputation of stability. The langauge also more rigidily enforces a functional style of programming. Though a, largely successful, drive to stick to the paradigm was a key criteria in the current version of the system, there are still certain areas where the code more resembles that seen in an imperative or object-oriented system, or where the code could simply be refactored to be much simpler. Using a more functional language would help improve this.
% section future_improvements (end)