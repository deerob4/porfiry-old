\section{Evaluation Critera}
When evaluating the system, several factors will to be taken into account to determine if the system itself, and the development process, is a success. They are listed below:

\begin{itemize}
\item Performance of the system should be considered. The system should be able to cope with at least 1000 connections at any one time, all communicating with each other. This communication should all be real time and seamless, and there should not be any lags or stutters caused directly by the system.

\item Staff and students should find the user interface simple and easy to use; they should instinctively know how to use the system, with a minimum amount of training required. This will be measured by observing users as they attempt to use the different sections of the system.

\item As the school requires the system relatively quickly, it is important that development of the system is completed within a reasonable period of time. To achieve this, it is important that time is spent only on features that the school have asked for, and not on ``useful'' extras.

\item The application code should be as high quality as possible - where possible, functional paradigms, such as using \textit{map} as opposed to a loop, should be followed, and code should be kept as modular and reusable as possible, to bring in all the advantages that such a development model provides.

\item The cost of the system to run should also be taken into account. Though the school will not be charged directly for the system, there will be costs relating to its continued maintenance and uptime. If these are overly expensive, there will be a negative impact on the success of the project.

\item Stability is an important factor, and it will also be considered. If the system is constantly prone to crashing, it could not be called stable; likewise, it would also indicate a failure of the test plan.

\item The impact of the system will also be considered. If the school decides that there are no benefits to using the system - in other words, it has failed to solve the problems laid out in the analysis - the system cannot be considered a success.

\item The suitability of the test plan will also be evaluated. A test plan is important because it provides a way of ensuring the parts of the system function correctly.

\item The system must meet all of the objectives, as follows:

\begin{itemize}
	\item Provide staff with an attractive user interface with which they can create, update and delete quizzes.
		\begin{itemize}
			\item Quizzes should allow for an unlimited number of multiple choice questions to be included, each of which should contain four possible answers - one of which is correct.
			\item Staff should be able to set how long they want to give students to answer the question, before the quiz moves on.
			\item Staff should have the ability to define an unlimited number of categories for the quiz, such as ``history'' or ``sport''; questions should be able to be placed in one of these categories.
			\item Staff should be able to specify the exact date and time that a quiz should begin.
		\end{itemize}

	\item Allow students of the school to answer the quiz in real time and compete against one another.
		\begin{itemize}
			\item The system should support up to 1000 concurrent students partaking in the quiz without failing.
			\item The quiz should begin at the specified time no matter how many users are connected. A five minute grace period should be given to take into account valid excuses for lateness.
			\item The students should be presented with a clear and attractive interface that displays the current category, the question, and the possible answers. It should show the remaining time that students have to choose an answer.
			\item When the time has elapsed for a question, the system should highlight the correct answer, and then immediately move onto the next question.
			\item Staff should have the ability to define an unlimited number of categories for the quiz, such as ``history'' or ``sport''; questions should be able to be placed in one of these categories.
			\item All of the above should take place simultaneously, on every individual screen.
			\item At the end of the quiz, the system should display the winning house, as well as the number of house points earned by each house.
		\end{itemize}

	\item Display a real time visualisation of how other participants in the quiz are answering.
		\begin{itemize}
			\item During the quiz, a pull out section should be available that displays the answers that other form groups in the school have chosen.
			\item If the student chooses to look at the panel before answering the question, they should only receive half a mark for their answer.
			\item All of the above should take place simultaneously, kept up to date on every individual screen.
		\end{itemize}

	\item Work effectively across a range of different devices and display types. The school plans to use the system across tablets and desktop computers. No matter which device a user is using, all aspects of the system should be available and fully functional.

	\item Theme itself to match the house colours of the logged in user. For example, the interface should be orange for those who belong to Baxter, green for Clive, blue for Acton, and so on.
\end{itemize}

\end{itemize}

If, after the completion of the project, these criteria are not met, it would be impossible, or at least very difficult, to label the system a complete success.
