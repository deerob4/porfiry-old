\subsection{Similar Systems}
There are a number of systems available, both free and at a cost, that would allow the school to improve their current method of quiztribution (\textit{quiz distribution}). Several popular options are outlined below.

\subsubsection{Quiz Creation Websites}
A number of websites exist that allow users to design, play and share their own quizzes. These websites, including \textit{QuizWorks}, \textit{ExamTime} and \textit{QuizBean} generally follow the same pattern: the user creates an account, is directed to an interface wherein they can design a quiz, and is then given a link with which they can share the quiz with others. For basic quiz creation, these websites are free, though for more advanced usage (\textit{QuizWorks} defines an ``advanced'' quiz as one containing more than 15 questions), paid plans are available.

As these systems are websites, they can be accessed from practically any computer or mobile device, as long as there is an internet connection in range. This means that users can continue to work on their quizzes, whether designing or answering them, outside of their place of work.

Though these systems are undoubtedly useful, and could, with a few compromises, be easily integrated into the school's routines, they lack an awareness of the structure of a school. There is no concept of ``form groups'' or ``heads of year'', both are which are vital concepts if the system is to meet what the school desires. Additionally, they lack the ability to display a detailed analysis of the results (at least, not without paying a somewhat exorbitant fee - \pounds60 per month in the case of \textit{QuizWorks}), a side effect of their focus on individuals as opposed to groups.

\subsubsection{Quiz Creation Software Packages}
Similar to quiz creation websites, quiz creation software packages allow the user to design and play a quiz. However, these systems are desktop applications (the majority are designed for Microsoft Windows), and so can only be accessed from a single desktop or laptop system. Examples of these systems include \textit{Wondershare Quiz Creator}, \textit{Tanida QuizBuilder}, and \textit{Articulate Storyline 2}. Unlike the mostly free websites, these software packages are often very expensive: the three systems mentioned range in price from \$99 - \$1846 for a single license, with additional licenses costing even more.

To compensate for the high prices, these desktop applications contain a vast feature set. Quizzes of every imaginable type can be created, from drag-and-drop, multiple choice, word bank quizzes, and many more. Images can be included, points assigned, and complex animations can be set to make the quiz as visually appealing as possible. In addition, reports can be generated with tremendous amounts of data, showcasing practically every data point imaginable.

Useful though these features are, they are a touch overkill for what the school's purposes. The systems are not the easiest things to use in the world, something that, considering the teacher's relative lack of IT skills, is quite a drawback. Additionally, the high costs make the systems prohibitively expensive, considering the school's status as the worst funded school in the county.
