\subsection{API Routes} % (fold)
\label{sub:api_routes}
This subsection lists the RESTful API routes - the section of the program that manages the database.
\subsubsection{server.js} % (fold)
This file bootstraps the backend server and loads all the different modules.
\lstinputlisting[language=javascript, caption=Bootstraps the backend.]{../app/server/server.js}
% subsubsection server_js (end)

\subsubsection{forms.js} % (fold)
\lstinputlisting[language=javascript, caption=Defines the resources available at /api/forms.]{../app/server/api/routes/forms.js}
% subsubsection server_js (end)

\subsubsection{quizzes.js} % (fold)
\lstinputlisting[language=javascript, caption=Defines the resources available at /api/quizzes.]{../app/server/api/routes/quizzes.js}
% subsubsection server_js (end)

\subsubsection{index.js} % (fold)
\lstinputlisting[language=javascript, caption=Exports API routes for easier importing.]{../app/server/api/index.js}
% subsubsection server_js (end)

% subsection api_routes (end)

\subsection{API Models} % (fold)
\label{sub:api_models}
This subsection lists the RESTful API resource models, defining exactly what a quiz, form, etc. look like.
\subsubsection{Quiz.js} % (fold)
\lstinputlisting[language=javascript, caption=Defines a quiz.]{../src/api/models/Quiz.js}
% subsubsection server_js (end)

\subsubsection{Category.js} % (fold)
\lstinputlisting[language=javascript, caption=Defines a category.]{../src/api/models/Category.js}
% subsubsection server_js (end)

\subsubsection{Question.js} % (fold)
\lstinputlisting[language=javascript, caption=Defines a question.]{../src/api/models/Question.js}
% subsubsection server_js (end)

\subsubsection{Answer.js} % (fold)
\lstinputlisting[language=javascript, caption=Defines an answer.]{../src/api/models/Answer.js}
% subsubsection server_js (end)

\subsubsection{Form.js} % (fold)
\lstinputlisting[language=javascript, caption=Defines a form group.]{../src/api/models/Form.js}
% subsubsection server_js (end)

% subsection api_models (end)

\subsection{Utility Functions} % (fold)
\label{sub:utility_functions}
This subsection lists the utility functions used through the system. These functions, often used multiple times in different places, are used to extract out common functionality into its own module.
\subsubsection{maxKey.js} % (fold)
\label{ssub:maxkey_js}
\lstinputlisting[language=javascript, caption=Returns the largest key in an object.]{../src/utils/maxKey.js}
% subsubsection maxkey_js (end)

\subsubsection{nextBiggest.js} % (fold)
\label{ssub:nextbiggest_js}
\lstinputlisting[language=javascript, caption=Returns the next largest value given an object.]{../src/utils/nextBiggest.js}
% subsubsection nextbiggest_js (end)

\subsubsection{choice.js} % (fold)
\label{ssub:choice_js}
\lstinputlisting[language=javascript, caption=Returns a random element from an array.]{../src/utils/choice.js}
% subsubsection choice_js (end)

\subsubsection{backgroundStyle.js} % (fold)
\label{ssub:backgroundstyle_js}
\lstinputlisting[language=javascript, caption=Returns the correct background style depending on the house.]{../src/utils/backgroundStyle.js}
% subsubsection backgroundstyle_js (end)

\subsubsection{settingsStyle.js} % (fold)
\label{ssub:settingsstyle_js}
\lstinputlisting[language=javascript, caption=Returns the correct settings styles depending on the house.]{../src/utils/settingsStyle.js}
% subsubsection settingsstyle_js (end)

\subsubsection{defaultQuiz.js} % (fold)
\label{ssub:defaultquiz_js}
\lstinputlisting[language=javascript, caption=Returns the default quiz, loaded into state whenever a new quiz is created.]{../src/utils/defaultQuiz.js}
% subsubsection defaultquiz_js (end)

\subsubsection{colourScheme.js} % (fold)
\label{ssub:colourscheme_js}
\lstinputlisting[language=javascript, caption=Returns a random colour scheme based on the user's house.]{../src/utils/colourScheme.js}
% subsubsection colourscheme_js (end)

% subsection utility_functions (end)

\subsection{Library Functions} % (fold)
\label{sub:library_functions}
This subsection lists the more advanced functions that are used. Focusing on aspects like scoring, these algorithms form the backbone of the processing that takes place for the quiz.
\subsubsection{housePoints.js} % (fold)
\label{ssub:housepoints_js}
\lstinputlisting[language=javascript, caption=Returns the number of house points earned from an answer.]{../src/libs/housePoints.js}
% subsubsection housepoints_js (end)

\subsubsection{answerStatistics.js} % (fold)
\label{ssub:answerstatistics_js}
\lstinputlisting[language=javascript, caption=Returns a tree of the most common answers in each form / year.]{../src/libs/answerStatistics.js}
% subsubsection answerstatistics_js (end)

\subsubsection{constructQuiz.js} % (fold)
\label{ssub:constructquiz_js}
\lstinputlisting[language=javascript, caption=Transforms the quiz made by the quiz creator into one suitable for consumption by the API.]{../src/libs/constructQuiz.js}
% subsubsection constructquiz_js (end)

% subsection library_functions (end)

\subsection{Action Creators} % (fold)
\label{sub:action_creators}
This subsection lists the action creators. These either construct an object that can be consumed by the reducers, or dispatch actions with side-effects, such as calling the API, which in turn affects an element in state.
\subsubsection{LoginActions.js} % (fold)
\label{ssub:loginactions_js}
\lstinputlisting[language=javascript, caption=Login actions string constants.]{../src/constants/LoginActions.js}
% subsubsection loginactions_js (end)

\subsubsection{login.js} % (fold)
\label{ssub:login_js}
\lstinputlisting[language=javascript, caption=Login action creators.]{../src/actions/login.js}
% subsubsection login_js (end)

\subsubsection{CreatorActions.js} % (fold)
\label{ssub:creatoractions_js}
\lstinputlisting[language=javascript, caption=Quiz creator actions string constants.]{../src/constants/CreatorActions.js}
% subsubsection creatoractions_js (end)

\subsubsection{maker.js} % (fold)
\label{ssub:maker_js}
\lstinputlisting[language=javascript, caption=Quiz creator action creators.]{../src/actions/maker.js}
% subsubsection maker_js (end)

% subsection action_creators (end)

\subsection{Reducers} % (fold)
\label{sub:reducers}
This subsections lists the application's state reducers.
\subsection{Reducers}
These processes will be used to manage the internal state of the application, storing information that the individual components need to function properly.

\subsubsection{Colour Reducer} % (fold)
\label{ssub:colour_reducer}
This reducer will control the colour theme, storing all the different sections in an object that can be passed down to all the application's sub-components.

\begin{verbatim}
process colours(state = call colourScheme(acton), action) {
  if (action = change colours) {
    replace colours with colours returned from colourScheme(action.house)
  }
}
\end{verbatim}
% subsubsection colour_reducer (end)

\subsubsection{User Reducer} % (fold)
\label{ssub:login_reducer}
This reducer will control the functions on the login screen, managing common state like the house and year of the user and the themed colours for the application.

\begin{verbatim}
process user(state = blank, action) {
  if (action = change house) {
    return the action's house
  }
  if (action = change year) {
    return the action's year
  }
  if (action = request quizzes) {
    return current state
    set requestingQuizzes to true
    set requestingQuizzesFailed to false
  }
  if (action = request quizzes failure) {
    return current state
    set requestingQuizzes to false
    set requestingQuizzesFailed to true
  }
  if (action = receive quizzes) {
    return current state
    set quizzes to action.quizzes
    set requestingQuizze to false
  }
  if (action = delete quiz success) {
    return current state
    filter out quizzes where quiz.id != action.id
  }
  if (action = quiz is ready) {
    return current state
    set quizIsReady to action.quizIsReady
  }
}
\end{verbatim}
% subsubsection login_reducer (end)

\subsubsection{Quiz Reducer} % (fold)
\label{ssub:quiz_reducer}
\begin{verbatim}
process settings(state = empty array, action) {
  if (action = update ID) {
    replace current ID with action ID
  }
  if (action = update title) {
    replace current title with action title
  }
  if (action = update start date) {
    replace current start date with action start date
  }
  if (action = update start time) {
    replace current start time with action start time
  }
  if (action = update question length) {
    replace current question length with action question length
  }
  if (action = update break length) {
    replace current break length with action break length
  }
  if (action = update is finished) {
    replace current is finished with action is finished
  }
  if (action = update all settings) {
    replace current all settings with action all settings
  }
}

process categories(state = empty array, action) {
  if (action = add category) {
    return new object combining current
    state and keys id and name set to the
    action properties
  }
  if (action = edit category) {
    map over each category in the array
    if category id = action.id replace the category name
    else add the category untouched
  }
  if (action = delete category) {
    filter out categories where category id = action id
  }
  if (action = delete all categories) {
    filter out categories where category id != -1 (all)
  }
}

process questions(state = empty array, action) {
  if (action = add question) {
    return new object combining current
    state keys, id, name and category id set to the
    action properties
  }
  if (action = edit question) {
    map over each question in the array
    if question id = action.id replace the question body
    else add the question untouched
  }
  if (action = delete question) {
    filter out questions where question id = action id
  }
  if (action = delete all questions) {
    filter out questions where question id != -1 (all)
  }
}

process answers(state = empty object, action) {
  if (action = add answer) {
    return new object combining current
    state keys, id, name, correct and question id set
    to the action properties
  }
  if (action = edit answer) {
    map over each answer in the array
    if answer id = action.id replace the answer name and
    correct keys else add the answer untouched
  }
  if (action = delete answer) {
    filter out answers where answer id = action id
  }
  if (action = delete all answers) {
    filter out answers where answer id != -1 (all)
  }
}
\end{verbatim}
% subsubsection quiz_reducer (end)

\subsubsection{Current Quiz Reducer} % (fold)
\label{ssub:current_quiz_reducer}
This reducer controls all the data needed for the running quiz, and allows it to be accessed by all the necessary sub-components.

\begin{verbatim}
process currentQuestion(state = 0, action) {
  if (action = show next answer) {
    set state.questionId to action.questionId
  }
}

process timeLeft(state = 10000, action) {
  if (action = decrement tine keft) {
    set state.timeLeft to action.timeLeft
  }
}

process inProgress(state = false, action) {
  if (action = begin quiz) {
    set quizInProgress to true
  }
  if (action = leave quiz) {
    set quizInProgress to false
  }
}

process answerStatistics(state = {
  acton: 0,
  baxter: 0,
  clive: 0,
  darwin: 0,
  houseman: 0,
  webb: 0
  }, action) {
    if (action = receive answer) {
      set state[answer.house] to answer.stats
    }
  }
\end{verbatim}
% subsubsection current_quiz_reducer (end)

% subsection reducers (end)

\subsection{Sockets} % (fold)
\label{sub:sockets}
This subsection lists the socket controllers used in the application. In order to allow the different clients (players) to communicate with one another, a concept known as web sockets is utilised. Web sockets allow a continual, bi-directional connection to be set up between each client and the server, facilitating the broadcasting and receiving of messages. In this manner, all of the multiplayer components to the quiz are controlled through these files.
\subsection{Sockets} % (fold)
\label{sub:sockets}
In order to power the real time nature aspects of the quiz, and keep every connected client in sync with the others, websockets will be used. The below processes outline exactly how they will be used.

\subsubsection{Question Timer} % (fold)
\label{ssub:question_timer}

% subsubsection question_timer (end)
% subsection sockets (end)

% subsection sockets (end)

\subsection{Components} % (fold)
\label{sub:reducers}
This subsections lists the application's components. Components make up the vast majority of the codebase, as they combine both markup - a description of what should be shown - and logic. The first three components are more advanced, as they interact directly with the applications's state. Other components are less intelligent, and rely on properties passed down from the more advanced components to perform actions and read data.
\subsubsection{LoginContainer.js} % (fold)
\label{ssub:login_js}
\lstinputlisting[language=javascript, caption=Advanced component controlling the login screen.]{../src/containers/LoginContainer.js}
% subsubsection login_js (end)

\subsubsection{LoginForm.js} % (fold)
\label{ssub:loginform_js}
\lstinputlisting[language=javascript, caption=Login form component.]{../src/components/LoginForm.js}
% subsubsection loginform_js (end)

\subsubsection{PlayQuiz.js} % (fold)
\label{ssub:playquiz_js}
\lstinputlisting[language=javascript, caption=Advanced component controlling the play quiz screen.]{../src/containers/PlayQuiz.js}
% subsubsection playquiz_js (end)

\subsubsection{CreateQuizContainer.js} % (fold)
\label{ssub:createquiz_js}
\lstinputlisting[language=javascript, caption=Advanced component controlling the create quiz screen.]{../src/containers/CreateQuizContainer.js}
% subsubsection createquiz_js (end)

\subsubsection{CreateQuiz.js} % (fold)
\label{ssub:createquiz_js}
\lstinputlisting[language=javascript, caption=Create quiz component]{../src/components/CreateQuiz.js}
% subsubsection createquiz_js (end)

\subsubsection{Button.js} % (fold)
\label{ssub:yearselector_js}
\lstinputlisting[language=javascript, caption=Button box component.]{../src/components/Button.js}
% subsubsection yearselector_js (end)

\subsubsection{Select.js} % (fold)
\label{ssub:yearselector_js}
\lstinputlisting[language=javascript, caption=Select box component.]{../src/components/Select.js}
% subsubsection yearselector_js (end)

% subsection reducers (end)

\subsection{Application Tooling} % (fold)
\label{sub:api_models}
This subsection lists the tooling and build process that is required for the application to run. Though not technically part of the source, a strong understanding is vital if one is to come to terms with the project.
\subsubsection{webpack.config.babel.js} % (fold)
This file compiles all the client dependencies for the application.
\lstinputlisting[language=javascript, caption=Bootstraps the backend.]{../webpack.config.js}
% subsubsection server_js (end)

\subsubsection{index.html} % (fold)
This file serves as an entry point for the client.
\lstinputlisting[language=html, caption=Main HTML entry.]{../index.html}
% subsubsection server_js (end)

% subsection api_models (end)

\subsection{Styles} % (fold)
\label{sub:styles}
This subsection lists the stylesheets that control the majority of the application's visual aspects.
\subsubsection{porfiry.scss} % (fold)
\label{ssub:porfiry_scss}
\lstinputlisting[language=css, caption=Entry point for styles, importing the other sheets.]{../src/styles/porfiry.scss}
% subsubsection porfiry_scss (end)

\subsubsection{typography.scss} % (fold)
\label{ssub:_typography_scss}
\lstinputlisting[language=css, caption=Styling for typographical elements of the application.]{../src/styles/base/_typography.scss}
% subsubsection _typography_scss (end)

\subsubsection{answer.scss} % (fold)
\label{ssub:answer_scss}
\lstinputlisting[language=css, caption=Styling for the answer blocks.]{../src/styles/components/_answer.scss}
% subsubsection answer_scss (end)

\subsubsection{button.scss} % (fold)
\label{ssub:button_scss}
\lstinputlisting[language=css, caption=Styling for the buttons.]{../src/styles/components/_button.scss}
% subsubsection button_scss (end)

\subsubsection{input.scss} % (fold)
\label{ssub:input_scss}
\lstinputlisting[language=css, caption=Styling for the text input elements.]{../src/styles/components/_input.scss}
% subsubsection input_scss (end)

\subsubsection{notification.scss} % (fold)
\label{ssub:notifications_scss}
\lstinputlisting[language=css, caption=Styling for the alert notifications.]{../src/styles/components/_notifications.scss}
% subsubsection notifications_scss (end)

\subsubsection{select.scss} % (fold)
\label{ssub:select_scss}
\lstinputlisting[language=css, caption=Styling for the select elements.]{../src/styles/components/_select.scss}
% subsubsection select_scss (end)

\subsubsection{login.scss} % (fold)
\label{ssub:login_scss}
\lstinputlisting[language=css, caption=Styling for the login form, positioning it properly.]{../src/styles/containers/login.scss}
% subsubsection login_scss (end)

\subsubsection{variables.scss} % (fold)
\label{ssub:variables_scss}
\lstinputlisting[language=css, caption=Variables used throughout the styles.]{../src/styles/helpers/_variables.scss}
% subsubsection variables_scss (end)

\subsubsection{mixins.scss} % (fold)
\label{ssub:mixins_scss}
\lstinputlisting[language=css, caption=Mixins used to apply processes across styles.]{../src/styles/helpers/_mixins.scss}
% subsubsection mixins_scss (end)

% subsection styles (end)
