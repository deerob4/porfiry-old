\subsubsection{Limitations of Current System} % (fold)
\label{ssub:limitations_of_current_system}
There are evidently a large number of issues with the above method. Firstly, distributing the quizzes via a word-processed document is not a particularly efficient method. It results in the network drive being cluttered with a variety of documents, perhaps with a non-existent naming scheme. This makes it harder for the form tutors to find the correct quiz for the week, slowing the whole process down. A more effective solution would be to have everything in it's own self contained system, with its own dedicated quiz screen, which points out the correct quiz to the tutors.

Additionally, having the questions and possible answers on the same document puts the integrity of the quiz at risk. Currently, form tutors get around this by hiding the document, but this can cause complications where students forget the possible answers, as well as other issues. It word be far more effective to always have the quiz displayed on screen on the interactive whiteboard, but the current system prohibits this.

Often, teachers forget about the quiz altogether, or believe it to be on a different date than when it is actually scheduled. This is a relatively common occurrence, and means that the quiz either has to be rescheduled (which those tutors who did remember find annoying), or that particular form has to miss out on the quiz that week; this can damage their overall reputation in the school community. A dedicated system could provide them a notification, perhaps via an email, that they should hold the quiz that afternoon. Additionally, the system could be set up in such a way that the quiz only begins once all the appropriate forms have connected.

Furthermore, the current system is not particularly fair. Students can spend as long as they wish on a single question, as long as the quiz is completed within the 25 minutes given to the form time. It would be fairer if the form was given a time limit of, say 60 seconds, after which the system automatically moves on to the next quiz.

The current system is also very isolated. Following the appointment of the new principal, the school has sought to implement the principle of ``togetherness'', whereby students work together more often. Though the current quiz system aligns itself with this philosophy to a degree (each form works together to come up with the answer), it could be improved by allowing a degree of interoperability between the forms. For example, if a quiz was being answered by all the forms in Year 8, one form could be given the opportunity to pose a question to the other forms, perhaps referencing one of the jokes sanctioned by the school.

By allowing the form tutors themselves to mark the quiz, their is a large risk of
inaccurate results being reported back, possibly altered in such a way that favours the form. Though this allows the head of year to display trust to his team of form tutors, there exists in the school a very competitive atmosphere, increasing the chance that such malpractice will occur. A safer approach would be to allow the system to mark the form's answers, and then report this directly to the head of year.

Though the heads of year throughout the school possess many fine and admirable qualities, it would be remiss to apply to them the label of ``mathematician''.  For simply calculating the best and worst performing forms for any given quiz, there are few issues with the current system (though it would be convenient if this was worked out automatically). It is when attempting to work out more complex results, such as the average score of a form over a period of several years, that the humble head of year falls short. A dedicated system would be able to perform a complicated analysis on the entire set of data it collects, allowing for a far more interesting report to be generated. This data could then be presented at an end of year, or even school, assembly, showcasing the best form in each category (or some other arbitrary statistic) throughout their entire school career.

Finally, the fact that the school is completely replacing the head of year system with new heads of house means that the entire approach is no longer possible.
% subsubsection limitations_of_current_system (end)
