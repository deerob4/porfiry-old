\section{Investigation of the Current System}
This section details the in-depth investigation that was performed on the current quiz system used by the school. Initial contact with the school was made via an email to one Mr Nick Bucknall, currently head of Year 8, on the 30th June 2015. Following a brief email exchange, the details of which can be found below, Below can be found the different aspects that were investigated.

% \subsection{Overview of System}
% Currently, the school does not have a formal, standardised system in place. A somewhat ad-hoc approach is currently used, using a combination of emails, documents scattered across the network, and verbal communication. Obviously, this approach has a number of issues, the details of which can be found below in the observations performed, as well as the analysis of the current system's limitations.

\subsection{Interviews}
A number of interviews were held with staff at the school. Three form tutors were interviewed, in order to gain an insight into how they believe form times could be improved through the use of interactive quizzes, and to get an idea of how they would like such a system to work. Additionally, Nick Bucknall, a head of year at the school, was interviewed; he was chosen in order to gain information on what results the system should calculate, and how best to integrate such a system within the school. He was also chosen to get a further idea of the faults with the current system. Full transcripts for each of the interviews are included below.

\subsubsection{Bryan Warr}

\textit{\textbf{DR:}} So what would you say is the main focus of form time?

\textit{\textbf{BW:}} Okay. So, there's several different types of form time. One could be focusing the students on silent reading, one could be on numeracy, one could be on literacy, another could be on a discussion - on something moral or spiritual. Obviously we've got assembly time as well, but it also depends on the audience - it depends on whether they're Year 7 or whether they're Year 11. So for Year 7's it would be different to Year 11's. For example, Year 11's might spend time revising, or focusing on exam technique; but Year 7's, a lot of time is spent with Year 7's building up the relationship with their tutor. So some of that could be activities that get them all working together.

\textit{\textbf{DR:}} I see. How important is that during the later years?

\textit{\textbf{BW:}} Yeah, it's still important, definitely. I suppose it depends on whether your tutor is new or not - you might have a tutor who's newer to Year 10 or Year 11, and wants to spend their time getting to know them. If you've had a tutor that's taken you all the way through, like my tutor group, then you don't need those ice breakers as much - you know each other quite well.

\textit{\textbf{DR:}} And how often does it occur that form teachers leave halfway through?

\textit{\textbf{BW:}} Here it's not very often. It's not as often as it is in other schools - in some schools it's all the time. My last school, I took a group through Year 11, and it was unusual, because every single tutor in that year group was with them from Year 7 to Year 11, which is very rare. But here for example, our Year 10 tutor group, pretty much similar tutor team from the start to the end. There's been a little change - so tutors do change, but there's always a set activity list from the head of year, head of house I should say.

\textit{\textbf{DR:}} So does the head of year have more control than the individual form tutors?

\textit{\textbf{BW:}} Yes. The form tutor is told by the head of year, well next year it will be the head of house, what to do. So we get given a timetable, it's on the wall, we get given a tutor program. Do you wanna see one?

\textit{\textbf{DR:}} Yeah, that would be helpful.

\begin{center}
\textit{\textbf{* DR and BW move to the classroom wall, to look at the timetable *}}
\end{center}

\textit{\textbf{BW:}} So these change every year. This is the Year 10 one - this is the summer one.

\begin{center}
\textit{\textbf{* Microphone fails to pick up conversation for the next 30 seconds. BW prints out tutor timetables for Year 9, Year 10 and Year 11, available below. BW and DR sit back down again. *}}\\
\end{center}

\textit{\textbf{BW:}} So you'll see that that one's key stage 3, and then you've got Year 10 and Year 11. So is you're idea maybe that you'd trial it here, or were you thinking...?

\textit{\textbf{DR:}} Yeah. The thing we've got to do is think of an idea for a system, and then go to a business - in my case it's here - and do interviews, with people who would use the system; hand out questionnaires, and perform observations. I've done my observation because I just used my experience here, and I'm doing interviews now.

\textit{\textbf{BW:}} So, Mr Bucknall would be a good one to talk to.

\textit{\textbf{DR:}} I've arranged one with him.

\textit{\textbf{BW:}} He's a head of house next year. So he's obviously, he's directly - he writes that. So the tutors, heads of house, who else? Also SLT, any members of SLT - I'm on SLT next year as well. Obviously the assemblies are run by members of SLT, but you can see there's a theme running through them - Year 9, you've got literacy, numeracy, in each one; you've got group discussions or activities, you've got silent reading in each one, because it just needs to be established. In Year 11, that would be revision, or personal study. And in Year 10 we give them that opportunity as well - we say, ``right, if you want to study, you do that.'' It's a lot more open than in Year 7.

\textit{\textbf{DR:}} The idea I had focuses mainly on the quiz aspect - automating it, so to speak, and making it more competitive. How do you feel about quizzes? I can see they're fairly regular on the timetable, but less so in Year 11.

\textit{\textbf{BW:}} Well, Miss Mitchell used to do them, and she was quite good at them. The quizzes are...I find them a bit hit and miss, on my own quizzes. The Year 10's do like them, my Year 10's. If they were automated, that would work. And if it was a competition between houses...

\textit{\textbf{DR:}} Yes, that's what I was thinking.

\textit{\textbf{BW:}} That's what we're...that would be really...I would have thought that would be quite well received, because obviously we're moving towards bigging up the house system next year. 

\textit{\textbf{DR:}} Okay.

\textit{\textbf{BW:}} So the house system is going to be more prevalent, so the fact that you're competing into forms...and also, heads of house would like it, because you've halved their workload, if a quiz is being laid on every week, definitely would be useful. It's obviously going to be difficult that you've got 7, 8, 9, 10 and 11...

\textit{\textbf{DR:}} All of whom need doing.

\textit{\textbf{BW:}} With different quizzes. That wouldn't be too difficult.

\textit{\textbf{DR:}} No. So, when I was in Year 11, Mr Bucknall seemed to control the quiz system quite heavily.

\textit{\textbf{BW:}} Yep.

\textit{\textbf{DR:}} So he'd make the quiz himself.

\textit{\textbf{BW:}} Yes, that's right.

\textit{\textbf{DR:}} And then handed out different quizzes. How do you feel about that, if that's what Mrs Shaw does with you?

\textit{\textbf{BW:}} Happy with that. If it was centralised, it would be even easier. So if the heads of house didn't have to handle that, if it was in a central system, that you could access, that would be easier.

\textit{\textbf{DR:}} So say there was an icon on the desktop, that opens up the program, showing the quiz for the week. 

\textit{\textbf{BW:}} Yes. Perfect. And if you chose Year 7, with a choice of the year groups, that would be easy. So the tutor could say ``right, I want the Year 7 one to come up'', that would be perfect. Yeah.

\textit{\textbf{DR:}} Because what I was thinking was the different heads of year, or houses now...if they wanted a quiz, they'd write it using the system, and then they'd have access to their form groups, so then they could target...and then they could send it that way. Would that be something they'd be interested in?

\textit{\textbf{BW:}} Possibly, but don't forget that now they're heads of house...so Mr Bucknall, who's got Darwin, he's now got five years. So again it's the same issue, each head of house has got to make lots of quizzes. 

\textit{\textbf{DR:}} So, there aren't heads of year anymore, there are houses?

\textit{\textbf{BW:}} Right. So we've got Acton, Baxter, Clive, Darwin, Houseman and Webb; and then each one's a house with a vertical setup to it.

\textit{\textbf{DR:}} I see.

\textit{\textbf{BW:}} So you're going to have the same issue with each one, in that they've got to differentiate between the years. And what they wouldn't want to do is replicate their work. So I don't think they'd be happy if they were...say Mrs Smith was doing this, and she had to do all the years, and Mrs Heath had to do the same, that would be a lot of repetition, whereas we could probably do the same...you could do the same thing for each one, couldn't you? If you did Year 7, just...bam. Year 8...right, just like that. That would be more sensible.

\textit{\textbf{DR:}} Okay. So, with assemblies and such, are the heads of houses taking each individual house each week? It wouldn't be say Mr Bucknall who speaks to the whole of Year 11 on a Tuesday?

\textit{\textbf{BW:}} Occasionally there will be a need for us to speak to Year 11 or Year 10 to do with certain aspects. But the majority of assemblies are with the houses, so it would be Mr Bucknall talking with Darwin - all the years. Okay? So it will be a house assembly. And we have whole school assemblies now on Mondays, first Monday of the month, which Mr Barratt does. So that's every Monday, and it's whole school, in the sports hall.

\textit{\textbf{DR:}} Kind of like it was at the end of term before?

\textit{\textbf{BW:}} Exactly, yeah, yeah. He's got it every month now, so we get together as a whole.

\textit{\textbf{DR:}} Right. So how do you feel...if the school's looking to improve cooperation between houses, how would you feel about that taking place in form time?

\textit{\textbf{BW:}} How do you mean?

\textit{\textbf{DR:}} Well, if, say, there was the quiz, say on that side of the screen, and on the other side was a view showing how the other forms are doing, in real time, would that be something you'd be interested in?

\textit{\textbf{BW:}} It could be interesting, yeah. Could be interesting. Definitely brings in the element of competition, live, rather...

\textit{\textbf{DR:}} Yeah, because that makes it a lot more interesting, than finding out say a week after that Baxter has won.

\textit{\textbf{BW:}} The only problem you've got is that they wouldn't necessarily be the same years on the same days.

\textit{\textbf{DR:}} How do you mean?

\textit{\textbf{BW:}} Well you'll have Acton...you're Year 7 quiz could be on the same day, but Year 7 and Year 10 might not be. Because the tutor programs are different, because they have to be, because of the assemblies and rooms, so for example the KS4 assembly there is on a Tuesday, but there it's silent reading. So they could be doing a quiz, but these lot could be in assembly. So there could be a house assembly on, when you've got Acton doing it. So, you'd have to look...

\textit{\textbf{DR:}} But horizontally, that wouldn't be an issue.

\textit{\textbf{BW:}} Right, horizontally, it would work, yeah. All of Year 7 would be together at the same time, unless there's a house assembly - if there's a house assembly on, it means that house would be out. So you'd have to look for where there's that literacy or numeracy activity, or something like a silent reading activity, where there's something with a group discussion. But I'm sure the heads of house wouldn't mind if, once a month, one of those...you'd have to ask Mr Bucknall of course, I'm sure he wouldn't mind...if one of those was taken up with a house quiz. Because if you did that, I'm just thinking...within a half term, you could get one quiz in for every year group. You could take one silent reading or one literacy or numeracy activity, and replace that with a house quiz, for the year group, that would work.

\textit{\textbf{DR:}} Okay. And what sort of topics would the quizzes best fit? Because when I did them, it kind of varied each week, so sometimes it would be culture, sometimes it would be general...stuff.

\textit{\textbf{BW:}} General knowledge is always good. Music's always good. Popular culture. Anything like that you could have. You could make it so they were subject related. So you could have history, geography or maths.

\textit{\textbf{DR:}} Right. So how would you, as head of maths, feel about there being a maths quiz?

\textit{\textbf{BW:}} Yeah. I'm happy with that. We're doing inter-house competitions within the subject, so we're going to do a Countdown activity. But I'd be absolutely fine with that, a maths quiz now and again, because again, it raises maths in the school. I'm sure the other heads of department are the same.

\textit{\textbf{DR:}} Okay. Trouble is trying to make it seem like a quiz, rather than a test, which I guess it would be.

\textit{\textbf{BW:}} Exactly. And also, you've gotta be careful with differentiation, because some students are much better at maths than others.

\textit{\textbf{DR:}} So would the top group have done things that the lower groups would have done?

\textit{\textbf{BW:}} Yes. Yeah, they'd have done some different things, so...maths quiz. See, my tutor group's a good example. I've got a tutor group where I've got some brilliant mathematicians, and some who really struggle, and to give them maths quizzes is always difficult, because some are at an advantage straight away - even with Countdown they're at an advantage. That would be a tricky one. General knowledge is probably better because everyone can access that. I mean, even if you did a History quiz, some people would be at an advantage. That's a tricky one. You could make it so that it would be subject related quiz, on a one off, where there was bits of Maths, bits of History. I suppose you could do that, make it more general.

\textit{\textbf{DR:}} Yeah, cause each form has people who are good at some things...

\textit{\textbf{BW:}} That might make it more accessible.

\textit{\textbf{DR:}} And how well would it work? If the form had to work with one another, to work out the answer?

\textit{\textbf{BW:}} That would be fine. Yeah, it wouldn't be an issue. The forms are used to working together.

\textit{\textbf{DR:}} And are all forms going to be happy working that way? Certain forms and year groups, I imagine, are less likely to work well with that sort of system.

\textit{\textbf{BW:}} Depends on the form tutor really. But the majority of the time, the form tutors accept what's asked of them. Certainly the majority of cases.

\textit{\textbf{DR:}} And how would participation be dealt with? There'd be some people at the back of the room who don't participate.

\textit{\textbf{BW:}} Right. You could do...to ensure participation, could be something like use the iPads. You could do something either...we do work using Socrative, using that. It's an app where everybody has to answer. So, I set the Socrative quiz up recently with my Year 7's. It's a web based one, so something like this would work. So with Socrative, I set a quiz up for the Year 7's, and you can see the live results come in. What they do is they can login to your room, and the quiz comes up as you're doing it - you get live results, with a load of reports. So I've got these results here, and these have all come through, and that's how it comes through. So there's the questions and answers - I can see that who's answered, and as they were doing it, I knew that Izzy hadn't done it. So maybe something like that would ensure participation.

\textit{\textbf{DR:}} So this is kind of like the old Quizdom system they used in Science.

\textit{\textbf{BW:}} Quizdom, exactly.

\textit{\textbf{DR:}} I never thought that worked very well.
 
\textit{\textbf{BW:}} No, these work a lot better - there's less faffing about. It's a web based system, so the room is accessible to anyone - it's constant; that will always be my room, whatever quiz I do. But that would work. Other than that, you're sort of relying on the teacher just...getting everybody involved.

\textit{\textbf{DR:}} Is that something the school is looking at - raising participation - or is that not really an issue?

\textit{\textbf{BW:}} We're looking at raising participation with the house system, and anything that does that is going to be useful. I don't think engagement in general is a problem in this school.

\textit{\textbf{DR:}} So if this sort of system were implemented in form time - purely in form time - would that be an issue, or should it be expanded for use in subjects?

\textit{\textbf{BW:}} I think it would be useful in form times to start with. Purely to address the problem the heads of house have. It also makes it more unilateral, doesn't it? Yeah, I think that would work.

\textit{\textbf{DR:}} How technical would you say that you are?

\textit{\textbf{BW:}} I'm pretty good. No, I'm not too bad. I can deal with most things, but not everybody's the same.

\textit{\textbf{DR:}} No. I'm trying to make it as simple as possible - thinking of people like Mrs England.

\textit{\textbf{BW:}} Yeah, some people would struggle. So it literally needs to be...two clicks. One click to get into the program, click to choose your year, maybe another to start the quiz.

\textit{\textbf{DR:}} How would you feel about a login system, where you had to sign in?

\textit{\textbf{BW:}} A login system would be fine. It would prevent the students hacking it.

\textit{\textbf{DR:}} Yeah, that's what I'm thinking.

\textit{\textbf{BW:}} Yeah, that would work. We're used to logging in to stuff. So yeah, happy with that.

\textit{\textbf{DR:}} This sounds a little crude, but would you be tempted to help your form in any with, if there was a reward for the best performing house?

\textit{\textbf{BW:}} Right. Are you suggesting would I cheat?

\textit{\textbf{DR:}} Yes.

\textit{\textbf{BW:}} Right. No, it's a good question. No, I wouldn't help them with it. I don't think they'd need my help - they work quite well together. I'm not convinced that every teacher would feel the same. \textit{\textbf{*laughs*}}

\textit{\textbf{DR:}} Okay. And is that going to be different depending on their year group?

\textit{\textbf{BW:}} Possibly, yeah. For Year 7 we'd approach it differently to Year 11.

\textit{\textbf{DR:}} And how suitable would this be if the school had, say, a theme of the term or theme of the week?

\textit{\textbf{BW:}} They do come up, but they would be more weekly things - so sometimes you have Anti-Bullying Week, or Global Entrepreneurship Week - so they come up now and again.

\textit{\textbf{DR:}} I see. And would this system come in useful for the quiz aspect of that?

\textit{\textbf{BW:}} Yeah, absolutely. And I'll tell you who else would be interested in this as well: Mrs Hancox and Miss Morris who run Life. Life would be really interested in this. I'm just thinking outside the box, purely because Friday lesson 2, when you have Life, you would have all of the year group in the same lesson at the same time - everybody in the school. So there might be an opportunity in Life as well.

\textit{\textbf{DR:}} I'll send them an email I think, thank you.

\textit{\textbf{BW:}} That might be worth talking about. You know...did Miss Hancox or Miss Morris teach you?

\textit{\textbf{DR:}} Miss Morris taught me for three years, yes.

\textit{\textbf{BW:}} Right, so you know her. So they're the heads of Life - they'd be interested in this, and that would be a good opportunity to use it, potentially. And with Life, they've got all sorts of things, so there's a lot of opportunity for quizzes.

\textit{\textbf{DR:}} If this were rolled out, how likely is it that it were actually used?

\textit{\textbf{BW:}} I think if people are told to use it, they'll use it.

\textit{\textbf{DR:}} Okay. And how would the heads of house tell you?

\textit{\textbf{BW:}} When I get this from Mrs Shaw, I follow that, because she tells me to. We know what we have to do - this on a Monday, that on a Wednesday. It might not be that we do the same thing, but we do some literacy and numeracy, then we do a group discussion or quiz, and then we have assembly. So we have that programme, and we follow it. We're actually checked up on whether we do that. SLT will check, randomly - we don't know when they're coming, but it's with the head of house, and they check what you're doing. If you're not doing what you're meant to...trouble.

\textit{\textbf{DR:}} Right. I remember last year, I had Mr Massey. And he wasn't...

\textit{\textbf{BW:}} \textit{\textbf{*laughs*}} You don't have to say any more!

\textit{\textbf{DR:}} And that was an issue sometimes.

\textit{\textbf{BW:}} Yeah. Okay. I feel your pain. \textit{\textbf{*laughs*}}

\textit{\textbf{DR:}} So, how would you feel about the quiz...

\textit{\textbf{BW:}} \textit{\textbf{*laughs*}}

\textit{\textbf{DR:}} About the quiz...\textit{\textbf{*laughs*}}...not starting until all the teachers had signed on?

\textit{\textbf{BW:}} Yeah, happy with that. I mean, it could be that you could trial it with one year group. You could ask the heads of house which year groups they feel most confident about giving it to. It might be that you've got a team of tutors in that year group who know they kids better, who've been together longer, who are maybe more confident dealing with this stuff, before you roll it out. Don't forget, you've got Year 7 tutors coming in September, who are gonna have to get to know their tutor groups. That'll be the priority for them. Year 11, pretty much starting them on exam technique...well, with the exception of your tutor \textit{\textbf{*laughs*}}. You want to get the ball rolling and get them into exam mode, so maybe a Year 9 team...\textit{\textbf{*laughs*}} yeah.

\textit{\textbf{DR:}} Sorry. \textit{\textbf{*laughs*}}

\textit{\textbf{BW:}} I dread to think! \textit{\textbf{*laughs*}} Who was in your tutor group? Name me some interesting characters.

\textit{\textbf{DR:}} Nick Porter.

\textit{\textbf{BW:}} Jesus. Right. Go on.

\textit{\textbf{DR:}} Doug Coull.

\textit{\textbf{BW:}} Oh right, not too bad.

\textit{\textbf{DR:}} Liam Davies.

\textit{\textbf{BW:}} \textit{\textbf{*laughs*}} Say no more - we're there!

\textit{\textbf{DR:}} \textit{\textbf{*laughs*}} So how would you feel about an email in the morning, reminding you of the quiz? Obviously you've got the timetable, but not all teachers are going to follow that.

\textit{\textbf{BW:}} \textit{\textbf{*laughs*}} Yeah, emails would work. Most of the teachers check their email in the morning.

\textit{\textbf{DR:}} That wouldn't be seen as too overbearing?

\textit{\textbf{BW:}} No, I don't think so. You could also set up a reminder on Outlook - that would work. Rather than an email, we could get a reminder, at whatever time the quiz is supposed to take place, so my Outlook is linked to my calendar. I get a reminder and it would just say ``Year 10 quiz - 1:40pm''. That might work as well.

\textit{\textbf{DR:}} Well, I think that's pretty much it.

\textit{\textbf{BW:}} Okay.

\textit{\textbf{DR:}} But thank you very much, you've been very helpful.

\textit{\textbf{BW:}} Anytime! I would set up a discussion with the heads of house - certainly Mr Bucknall.

\subsubsection{Wendy Blower}

\textit{Interview scheduled for 7th July.}

\subsubsection{Carla Hancox and Rachel Morris}

\subsubsection{Nick Bucknall}

\textit{Interview scheduled for 14th July.}

\subsubsection{Tim Goodman}

\subsection{Questionnaires}
An online questionnaire was created for students of the school to fill in, detailing their opinion on form times, how it could be improved through the use of quizzes.

\subsubsection{Questions}
The following questions were included	 on the questionnaire, distributed to students at the school:

\noindent\fbox{%
    \parbox{\textwidth}{%
        \begin{enumerate}[leftmargin=0cm,itemindent=.5cm,labelwidth=\itemindent,labelsep=0cm,align=left]
					\item How satisfied are you with your current form time experience? 1 2 3 4\\
					\item ``My form times are always well structured.'' How far do you agree or disagree with this statement? 1 2 3 4 5\\
					\item Out of the following activities you selected, which do you enjoy the most?
								\begin{itemize}
									\item Silent reading
									\item Knowledge quiz
									\item Group discussion
									\item Board games
									\item Physical activities
									\item Other\\
								\end{itemize}
					\item Approximately how often do quizzes feature in your form times? 1 2 3 4\\
					\item ``My form times would be improved if we did quizzes more often.'' How far do you agree or disagree with this statement? 1 2 3 4\\
					\item ``I find that quizzes improve the relationship between me and the rest of my form'' How far do you agree or disagree with this statement? 1 2 3 4\\
					\item Which topics have you been given quizzes on? Select all that apply.
								\begin{itemize}
									\item General knowledge
									\item Media
									\item Sports
									\item History
									\item Literature
									\item Relationships
									\item Brainteasers
									\item Other\\
								\end{itemize}
					\item To what extent do you interact with other forms in your year during form time? 1 2 3 4\\
					\item Do you believe form times would be improved by interaction with other forms? Yes No\\
					\item If you wish, please expand on your answer to the previous question.
				\end{enumerate}
    }%
}

\subsubsection{Results}
The following results were collected after the quiz had been live for 10 days. They were automatically generated using the 

\subsection{Observations}
Having been a member of the school community for over five years, I am well placed to provide an observation on how the school currently goes about creating, setting and analysing quizzes used in form times. Currently, no formal system is in place; an ad-hoc system is used, following this general pattern:

\begin{enumerate}
	\item The head of year creating the quiz thinks of a set of questions and possible answers, usually following a theme, and then writes them down on a Microsoft Word document. The correct answer is marked out, to aid the form tutor in marking the quiz. This document is then saved to a drive on the school's LAN.

	\item The head of year then notifies the individual form tutors of the quiz, usually at one of their weekly meetings, and tells them to conduct the quiz with their form group on a certain date.

	\item When the date is reached, the form tutor opens the document from the network, ensuring that the document is kept hidden to avoid members of the form viewing the correct answers.

	\item The form tutor reads each question out in turn, and the members of the form work together to attempt to work out the answer. They either come up with their own answer or choose from a list of options, depending on whether or not the question is multiple choice. The form tutor marks down the answer they chose, and this process repeats until the quiz is completed.

	\item Once all the questions have been answered, the form tutor adds up the total number of marks achieved by the form.

	\item The form tutor then passes the mark onto the head of year, either by email or when passing them in the corridor or the staffroom. This task is sometimes performed by a member of the form themselves, occasionally with the expectation that the mark achieved will be exaggerated somewhat.

	\item After receiving all the results, the head of year works out which form achieved the highest result, and which the lowest. This result is reported back to the year in the weekly assembly, often with a small reward for the highest achieving form.
\end{enumerate}

\subsection{Document Inspections}

\subsection{Similar Systems}
There are a number of systems available, both free and at a cost, that would allow the school to improve their current method of quiztribution (\textit{quiz distribution}). Several popular options are outlined below.

\subsubsection{Quiz Creation Websites}
A number of websites exist that allow users to design, play and share their own quizzes. These websites, including \textit{QuizWorks}, \textit{ExamTime} and \textit{QuizBean} generally follow the same pattern: the user creates an account, is directed to an interface wherein they can design a quiz, and is then given a link with which they can share the quiz with others. For basic quiz creation, these websites are free, though for more advanced usage (\textit{QuizWorks} defines an ``advanced'' quiz as one containing more than 15 questions), paid plans are available.

As these systems are websites, they can be accessed from practically any computer or mobile device, as long as there is an internet connection in range. This means that users can continue to work on their quizzes, whether designing or answering them, outside of their place of work.

Though these systems are undoubtedly useful, and could, with a few compromises, be easily integrated into the school's routines, they lack an awareness of the structure of a school. There is no concept of ``form groups'' or ``heads of year'', both are which are vital concepts if the system is to meet what the school desires. Additionally, they lack the ability to display a detailed analysis of the results (at least, not without paying a somewhat exorbitant fee - \pounds60 per month in the case of \textit{QuizWorks}), a side effect of their focus on individuals as opposed to groups.

\subsubsection{Quiz Creation Software Packages}
Similar to quiz creation websites, quiz creation software packages allow the user to design and play a quiz. However, these systems are desktop applications (the majority are designed for Microsoft Windows), and so can only be accessed from a single desktop or laptop system. Examples of these systems include \textit{Wondershare Quiz Creator}, \textit{Tanida QuizBuilder}, and \textit{Articulate Storyline 2}. Unlike the mostly free websites, these software packages are often very expensive: the three systems mentioned range in price from \$99 - \$1846 for a single license, with additional licenses costing even more.

To compensate for the high prices, these desktop applications contain a vast feature set. Quizzes of every imaginable type can be created, from drag-and-drop, multiple choice, word bank quizzes, and many more. Images can be included, points assigned, and complex animations can be set to make the quiz as visually appealing as possible. In addition, reports can be generated with tremendous amounts of data, showcasing practically every data point imaginable.

Useful though these features are, they are a touch overkill for what the school's purposes. The systems are not the easiest things to use in the world, something that, considering the teacher's relative lack of IT skills, is quite a drawback. Additionally, the high costs make the systems prohibitively expensive, considering the school's status as the worst funded school in the county.

\subsubsection{Quizdom}


\subsection{Justification of Methods}

\subsection{IPSO Chart}
This IPSO chart describes the inputs, outputs, storage locations and general processes that are associated with the system.

\begin{table}[h]
\centering
\begin{tabular}{|l|l|}
\hline
\multicolumn{1}{|c|}{{\bf Inputs}}                                                                                                                              & \multicolumn{1}{c|}{{\bf Processes}}                                                                                                                                                                                                                                                                         \\ \hline
\begin{tabular}[c]{@{}l@{}}Full name\\ Password\\ \\ Form name\\ \\ Quiz title\\ Quiz questions\\ Quiz answers\\ Forms the quiz should be sent to.\end{tabular} & \begin{tabular}[c]{@{}l@{}}Create HOY accounts from inputs\\ Create form group accounts\\ \\ Create and store quizzes\\ \\ Allow quizzes to be answered by\\ forms and store answers.\\ \\ Prevent quizzes from starting unless\\ all forms have joined.\\ \\ Mark quizzes and analyse results.\end{tabular} \\ \hline
\multicolumn{1}{|c|}{{\bf Storage}}                                                                                                                             & \multicolumn{1}{c|}{{\bf Outputs}}                                                                                                                                                                                                                                                                           \\ \hline
\begin{tabular}[c]{@{}l@{}}Store user accounts\\ Store individual quizzes with answers.\\ Store mark for quizzes.\\ Store analysis of results.\end{tabular}     & \begin{tabular}[c]{@{}l@{}}Quiz containing questions and answers.\\ Analysis of quiz results.\end{tabular}                                                                                                                                                                                                   \\ \hline
\end{tabular}
\caption{IPSO Chart}
\label{my-label}
\end{table}

\subsection{Limitations of Current System}
There are evidently a large number of issues with the above method. Firstly, distributing the quizzes via a word-processed document is not a particularly efficient method. It results in the network drive being cluttered with a variety of documents, perhaps with a non-existent naming scheme. This makes it harder for the form tutors to find the correct quiz for the week, slowing the whole process down. A more effective solution would be to have everything in it's own self contained system, with its own dedicated quiz screen, which points out the correct quiz to the tutors.

Additionally, having the questions and possible answers on the same document puts the integrity of the quiz at risk. Currently, form tutors get around this by hiding the document, but this can cause complications where students forget the possible answers, as well as other issues. It word be far more effective to always have the quiz displayed on screen on the interactive whiteboard, but the current system prohibits this.

Often, teachers forget about the quiz altogether, or believe it to be on a different date than when it is actually scheduled. This is a relatively common occurrence, and means that the quiz either has to be rescheduled (which those tutors who did remember find annoying), or that particular form has to miss out on the quiz that week; this can damage their overall reputation in the school community. A dedicated system could provide them a notification, perhaps via an email, that they should hold the quiz that afternoon. Additionally, the system could be set up in such a way that the quiz only begins once all the appropriate forms have connected.

Furthermore, the current system is not particularly fair. Students can spend as long as they wish on a single question, as long as the quiz is completed within the 25 minutes given to the form time. It would be fairer if the form was given a time limit of, say 60 seconds, after which the system automatically moves on to the next quiz.

The current system is also very isolated. Following the appointment of the new principal, the school has sought to implement the principle of ``togetherness'', whereby students work together more often. Though the current quiz system aligns itself with this philosophy to a degree (each form works together to come up with the answer), it could be improved by allowing a degree of interoperability between the forms. For example, if a quiz was being answered by all the forms in Year 8, one form could be given the opportunity to pose a question to the other forms, perhaps referencing one of the jokes sanctioned by the school.

By allowing the form tutors themselves to mark the quiz, their is a large risk of
inaccurate results being reported back, possibly altered in such a way that favours the form. Though this allows the head of year to display trust to his team of form tutors, there exists in the school a very competitive atmosphere, increasing the chance that such malpractice will occur. A safer approach would be to allow the system to mark the form's answers, and then report this directly to the head of year.

Though the heads of year throughout the school possess many fine and admirable qualities, it would be remiss to apply to them the label of ``mathematician''.  For simply calculating the best and worst performing forms for any given quiz, there are few issues with the current system (though it would be convenient if this was worked out automatically). It is when attempting to work out more complex results, such as the average score of a form over a period of several years, that the humble head of year falls short. A dedicated system would be able to perform a complicated analysis on the entire set of data it collects, allowing for a far more interesting report to be generated. This data could then be presented at an end of year, or even school, assembly, showcasing the best form in each category (or some other arbitrary statistic) throughout their entire school career.

Finally, the fact that the school is completely replacing the head of year system with new heads of house means that the entire approach