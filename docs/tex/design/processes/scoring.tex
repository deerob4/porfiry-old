\subsection{Scoring}
Much of the application will focus on scoring the quiz, using the answer packets that each answer to a question emits.

\subsubsection{Calculate House Points}
This function will calculate the number of house points that an answer packet generates, and then compute an up to date list of the current score for either each year or house.
\begin{verbatim}
process housePoints(packet, correct, keys, prop, state = {}) {
  if state is empty {
    loop through each key
    set state[key] = 0
  }

  if answer is correct and peek is false {
    set score = 1
  } else if answer is correct and peek is true {
    set score = 0.5
  } else set score = 0

  return clone of state
  set packet[prop] to state[packet[prop]] add score
}
\end{verbatim}
By taking a property value as an argument, the function can be used to calculate house points for any specified group, making it more versatile. Additionally, by only computing the most recent state for a single answer packet, and using the previous state as a starting point, it becomes easier to apply the function in the real time scenario it will be used in - an average of 22,500 calls per quiz, many happening at once.

\subsubsection{Highest Value in Object}
This function will calculate the highest value in a given object. It will be used in the most common answer function, displayed below.
\begin{verbatim}
function highest(object) {
  set variable keys to array of object keys
  set variable vals to map over object using keys variable for an array of object values
  call max function from standard library on variable vals
}
\end{verbatim}

\subsubsection{Correct Answer}
Using an answer packet, returns true if the user chose the correct answer.
\begin{verbatim}
function isCorrect(packet, correctAnswer) {
  if packet.answer = correctAnswer {
    Highlight box green.
  }
  otherwise highlight box red.
}
\end{verbatim}

\subsubsection{Maximum Keys}
This function will take an object where every key is a number. It will return the key of the highest number.
process maxKey(object) {
  perform reduce on object ->
  if obj[a] > obj[b] return a
  else return b
}

\subsubsection{Most Common Answer}
Using the answere packet, this function will work out the most common answers for each question at any one point in the quiz. It works in much the same way as the function for calculating house points, but the appending operations differ so a new function is needed.
\begin{verbatim}
process commonAnswer(packet, keys, state) {
  set variable ``inner'' to object containing ``mode'' set to undefined and each house, set to 0
  if the ``state'' is empty, extend ``state'' with ``inner''
  set ``newState'' to a clone of ``state'', but set the answer frequency for the house the user chose
  to itself + 1
  set the ``mode'' property on ``newState'' to the maximum keys of house freqs.
  return ``newState''
}
\end{verbatim}
