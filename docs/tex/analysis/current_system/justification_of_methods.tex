\subsection{Justification of Methods} % (fold)
\label{sub:justification_of_methods}
As is custom in such situations, a range of investigatory methods have been used to gain a deep insight into the current methods used by the school in their management of quizzes. By using a range of methods, including interviews, document inspections and observations, it is possible to uncover a wide range of information relating to the topic, all of which can be used to develop a suitable range of objectives that result in a system that meets the school's requirements.

\subsubsection{Interviews} % (fold)
\label{ssub:interviews}
The most useful method of investigation used was interviews. Specific members of staff were chosen for interview, ranging from normal teachers, to more senior figures, to the headmaster himself. By doing this, it was possible to gain an appreciation for how staff would like an automated quiz system to work, and how it could fit into the school structure. This has allowed a set of objectives to be created that fully meet the school's demands. Additionally, the chance to interview a student provided additional information on what would make the system suitable. Through these interviews, therefore, all users of the system were covered. Interviews are particularly suited to this sort of task because they allow for a real conversation to be developed (as can be seen in the above transcripts). The interviewee can be asked to expand on interesting points, ensuring that as much information as possible is gleaned from the session. Of course, some members of staff would be uncomfortable with such a direct meeting (some use teaching styles that are best described as indirect), but all of the staff chosen for these interviews were perfectly willing to speak, and a number of spoke candidly and honestly about their opinions regarding the management of the school and how this could impact the success of the project.
% subsubsection interviews (end)

\subsubsection{Questionnaires} % (fold)
\label{ssub:questionnaires}
Though questionnaires are very useful and can provide a wide range of data, it was felt that, for this investigation, they would not be appropriate. Part of this comes down to the fact that other methods investigation are enough: interviews, document inspections and observations all provide a wealth of information, and there is little that a questionnaire could provide that the other methods could not. The second issue comes down to the policies in place within the school regarding questionnaires from external parties. Though the cause is a good one, and the questionnaire would originate from a trusted source, the school has a policy of declining questionnaire requests, to prevent students being exposed to ideas not in keeping with those officially sanctioned by the leadership team.
% subsubsection questionnaires (end) subsection justification_of_methods (end)

\subsubsection{Observations} % (fold)
\label{ssub:observations}
Due to many years of experience with the system, it was felt that no formal observation was required. However, these years of experience have resulted in a deep familiarity of the current solution, providing an accurate picture of its advantages and disadvatages. These can then be applied to the new system, in order to produce a product that builds on the good aspects of the current system, whilst building on areas that require solution. As a result of the informal nature of the observation (indeed, neither myself nor the teaching staff knew it was going on) there was no chance of bias, therefore avoiding some of the pitfalls common in observations, such as a non-standard routine being followed in order to provide a less than honest view of events to the observer.
% subsubsection observations (end)

\subsubsection{Document Inspections} % (fold)
\label{ssub:document_inspections}
Whilst, at first glance, the inspection of documents related to a system is perhaps not as useful as other, potentially more active methods, such as observations or interviews, there is no doubt that they do serve a purpose. In this case, inspecting documents led to an understanding as to the suitability of the new system, and the ease with which it could be integrated into the school environment. Additionally, inspecting documents served as a supplement to the research gathered during the interviews and observations, providing hard evidence of a sort that backs up statements made in these sections. If the analysed system were more dependent on documents, then this method of investigation would likely have come in more useful.
% subsubsection document_inspections (end)