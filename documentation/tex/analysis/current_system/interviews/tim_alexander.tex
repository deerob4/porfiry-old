Alexander was questioned over the strength of the school's Wi-Fi network, and it's ability to cope with a very large number of simultaneous connections, all sending and receiving data. He replied that the network should easily be able to cope with such a high demand, as the school had recently upgraded its network infrastructure to a quality comparable to that found in the enterprise. The network interface was brought up on a nearby screen, displaying the usual number of connections and how this affects the network. A forecast feature was also shown, indicating that the network will be able to cope with up to 5000 simultaneous connections at once before an upgrade is needed. Alexander also stated that almost all of the school site - including every classroom and even a decent way across the football pitch - has access to the network.

Secondly, a discussion was held over the school's iPad policy, and the feasibility of pushing an application to all of the iPads. Alexander stated that, though in most years the majority of students are in possession and make use of an iPad, most of these are their own, and so cannot be accessed by the school. Through his knowledge of deployment methods, Alexander revealed that, even if the school was able to push the application to all of the students, the application would have to published in Apple's App Store. This would cost a fee of \pounds99 a year, which Alexander stated the school would be unwilling to pay, and would also result in the application having to be listed publicly; this would not be a good idea, for obvious reasons. Following this revelation, Alexander advised that the wisest method would be to develop the system purely as a web application, and not worry on packaging it up for distribution.

Following this, Alexander was asked about the feasibility of deploying the system source itself directly on to the school system. Many words were spoken on this topic, and the result can be summarised in one (admittedly rather long) sentence: though technically possible, it would be rather difficult to setup; the system in place at the school is not 100\% compatible with the methods proposed to build the system, and this could impact the quality; additionally, it could represent a security risk if the application were placed too near to the school's student database; though this could be mitigated by using a virtual machine, it would result in some licensing issues; to summarise, the entire endeavour would almost certainly fail. Instead Alexander recommended that the system be deployed on a hosted cloud provider such as Amazon EC2 or Heroku. Though this would cost money, the school would be happy to pay for this; a visit to a man named Duncan, ostensibly the school's finance manager, confirmed this.