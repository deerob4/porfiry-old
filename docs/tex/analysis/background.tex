\section{Background}

\subsection{About the Business}

The Priory School is a medium sized secondary school located in Shrewsbury, Shropshire. Employing over 100 teaching staff, there are approximately 900 pupils on roll, ranging in age from 11 - 16. Each belongs to an individual form group. During it's previous two inspections, Ofsted judged the school to be Outstanding, the highest possible rating. The school is a founding member of the Salop Teaching Alliance, giving them high standing in the local area. Additionally, the school has the highest attendance rate in the county, and achieves exam results well above the national average. 

Like many schools, a house system is in place, which is linked to the form system. Each student belongs to one of five houses, either Acton, Baxter, Clive, Houseman, Darwin or Webb (named after famous local figures). This house is then combined with the student's year to produce their form group - a Year 7 student who belongs to Baxter is in the form 7B, a Year 8 student in Houseman belongs to 8H, and so on. Each of these forms is led by a form tutor, a member of staff who calls the register, provides support, and generally acts as the first point of contact for a student.

A change of leadership in January 2015 resulted in the previous headteacher, Ms Candy Garbett, leaving the school; Mr Michael Barratt, previously of Adams Grammar School, Newport, became the new principal. Following this change in leadership, the school has sought to embrace the advantages of technology, and has invested in several new systems, including an online homework tracker, a virtual learning environment, and a library tracking system. This newfound acceptance of technology opens the way for this project.

Previously, interaction between forms was focused on year groups: students in 7A and 7B would be encouraged to spend time with one another, but getting together within houses - students from 7A and 9A, for example - 

However, due to Michael Barratt's experience in leading a grammar school, and his subsequent fondness for the more traditional house system, this horizontal approach has now been modified to a vertical one: each previous head of year is now known as a head of house; and, rather than controlling an individual year group, they now control an individual house. For example, Nick Bucknall has been placed in charge of Acton, and so controls 7A - 11A.

\subsection{About the Project}
Like many schools, The Priory School makes use of a form time in the afternoon. During this process, students are registered, bulletins are read out, and a timetabled activity is carried out; these activities usually include silent reading, a group debate, and quizzes. These quizzes are usually designed by the head of year, and include a range of topics, from current affairs, educational matters, and simple trivia.

Currently, these quizzes are delivered to the forms on a Microsoft Word document, via the school's LAN. The members of the form work together to arrive at what they believe to be the correct answer, and once all questions have been answered, the form tutor marks the quiz and returns the result to the head of year, usually orally. This will therefore aid in following the school's new policy of ``togetherness''.
