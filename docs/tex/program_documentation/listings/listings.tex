\subsection{API Routes} % (fold)
\label{sub:api_routes}
This subsection lists the RESTful API routes - the section of the program that manages the database.
\subsubsection{API Endpoints}
The following table is a plan of the API endpoints that will be used to receive some of the less volatile data, containing the HTTP method used to access the route, the route itself, any parameters used, and an explanation of what the endpoint will return.

\begin{table}[]
\centering
\caption{My caption}
\label{my-label}
\begin{tabular}{|l|l|l|l|}
\hline
\multicolumn{1}{|c|}{{\bf Method}} & \multicolumn{1}{c|}{{\bf Route}}     & \multicolumn{1}{c|}{{\bf Params}} & \multicolumn{1}{c|}{{\bf Role}} \\ \hline
GET                                & /api/quizzes                         & null                              & Return all the quizzes.         \\ \hline
GET                                & /api/quizzes/\textlessid\textgreater & id                                & Return a specific quiz.         \\ \hline
POST                               & /api/quizzes                         & null                              & Add a new quiz.                 \\ \hline
PUT                                & /api/quizzes/\textlessid\textgreater & id                                & Update a specific quiz.         \\ \hline
GET                                & /api/forms                           & null                              & Return all the forms.           \\ \hline
GET                                & /api/forms/\textlessid\textgreater   & id                                & Return a specific form.         \\ \hline
POST                               & /api/forms                           & null                              & Add a new form.                 \\ \hline
PUT                                & /api/forms/\textlessid\textgreater   & id                                & Update a specific form.         \\ \hline
\end{tabular}
\end{table}

% subsection api_routes (end)

\subsection{API Models} % (fold)
\label{sub:api_models}
This subsection lists the RESTful API resource models, defining exactly what a quiz, form, etc. look like.
\subsubsection{Quiz.js} % (fold)
\lstinputlisting[language=javascript, caption=Defines a quiz.]{../src/api/models/Quiz.js}
% subsubsection server_js (end)

\subsubsection{Category.js} % (fold)
\lstinputlisting[language=javascript, caption=Defines a category.]{../src/api/models/Category.js}
% subsubsection server_js (end)

\subsubsection{Question.js} % (fold)
\lstinputlisting[language=javascript, caption=Defines a question.]{../src/api/models/Question.js}
% subsubsection server_js (end)

\subsubsection{Answer.js} % (fold)
\lstinputlisting[language=javascript, caption=Defines an answer.]{../src/api/models/Answer.js}
% subsubsection server_js (end)

% subsection api_models (end)

\subsection{Utility Functions} % (fold)
\label{sub:utility_functions}
This subsection lists the utility functions used through the system. These functions, often used multiple times in different places, are used to extract out common functionality into its own module.
\subsubsection{maxKey.js} % (fold)
\label{ssub:maxkey_js}
\lstinputlisting[language=javascript, caption=Returns the largest key in an object.]{../src/utils/maxKey.js}
% subsubsection maxkey_js (end)

\subsubsection{nextBiggest.js} % (fold)
\label{ssub:nextbiggest_js}
\lstinputlisting[language=javascript, caption=Returns the next largest value given an object.]{../src/utils/nextBiggest.js}
% subsubsection nextbiggest_js (end)

\subsubsection{choice.js} % (fold)
\label{ssub:choice_js}
\lstinputlisting[language=javascript, caption=Returns a random element from an array.]{../src/utils/choice.js}
% subsubsection choice_js (end)

\subsubsection{backgroundStyle.js} % (fold)
\label{ssub:backgroundstyle_js}
\lstinputlisting[language=javascript, caption=Returns the correct background style depending on the house.]{../src/utils/backgroundStyle.js}
% subsubsection backgroundstyle_js (end)

\subsubsection{settingsStyle.js} % (fold)
\label{ssub:settingsstyle_js}
\lstinputlisting[language=javascript, caption=Returns the correct settings styles depending on the house.]{../src/utils/settingsStyle.js}
% subsubsection settingsstyle_js (end)

% subsection utility_functions (end)

\subsection{Library Functions} % (fold)
\label{sub:library_functions}
This subsection lists the more advanced functions that are used. Focusing on aspects like scoring, these algorithms form the backbone of the processing that takes place for the quiz.
\subsubsection{housePoints.js} % (fold)
\label{ssub:housepoints_js}
\lstinputlisting[language=javascript, caption=Returns the number of house points earned from an answer.]{../src/libs/housePoints.js}
% subsubsection housepoints_js (end)

\subsubsection{answerStatistics.js} % (fold)
\label{ssub:answerstatistics_js}
\lstinputlisting[language=javascript, caption=Returns a tree of the most common answers in each form / year.]{../src/libs/answerStatistics.js}
% subsubsection answerstatistics_js (end)

\subsubsection{constructQuiz.js} % (fold)
\label{ssub:constructquiz_js}
\lstinputlisting[language=javascript, caption=Transforms the quiz made by the quiz creator into one suitable for consumption by the API.]{../src/libs/constructQuiz.js}
% subsubsection constructquiz_js (end)

\subsubsection{flattenQuiz.js} % (fold)
\label{ssub:flattenquiz_js}
\lstinputlisting[language=javascript, caption=Normalises the quiz returned from the API so it can be read by the internal state.]{../src/libs/flattenQuiz.js}
% subsubsection flattenquiz_js (end)

% subsection library_functions (end)

\subsection{Action Creators} % (fold)
\label{sub:action_creators}
This subsection lists the action creators. These either construct an object that can be consumed by the reducers, or dispatch actions with side-effects, such as calling the API, which in turn affects an element in state.
\subsubsection{LoginActions.js} % (fold)
\label{ssub:loginactions_js}
\lstinputlisting[language=javascript, caption=Login actions string constants.]{../src/constants/LoginActions.js}
% subsubsection loginactions_js (end)

\subsubsection{login.js} % (fold)
\label{ssub:login_js}
\lstinputlisting[language=javascript, caption=Login action creators.]{../src/actions/login.js}
% subsubsection login_js (end)

\subsubsection{CreatorActions.js} % (fold)
\label{ssub:creatoractions_js}
\lstinputlisting[language=javascript, caption=Quiz creator actions string constants.]{../src/constants/CreatorActions.js}
% subsubsection creatoractions_js (end)

\subsubsection{maker.js} % (fold)
\label{ssub:maker_js}
\lstinputlisting[language=javascript, caption=Quiz creator action creators.]{../src/actions/maker.js}
% subsubsection maker_js (end)

% subsection action_creators (end)

\subsection{Reducers} % (fold)
\label{sub:reducers}
This subsections lists the application's state reducers.
\subsubsection{configureStore.js} % (fold)
\label{ssub:configurestore_js}
\lstinputlisting[language=javascript, caption=Creates the store containing application state and initialises middleware.]{../src/store/configureStore.js}
% subsubsection configurestore_js (end)

\subsubsection{colourReducer.js} % (fold)
\label{ssub:colourreducer_js}
\lstinputlisting[language=javascript, caption=Colour reducer controlling the application's current colours (based on house).]{../src/reducers/colourReducer.js}
% subsubsection colourreducer_js (end)

\subsubsection{userReducer.js} % (fold)
\label{ssub:userreducer_js}
\lstinputlisting[language=javascript, caption=User reducer controlling creation state.]{../src/reducers/userReducer.js}
% subsubsection userreducer_js (end)

\subsubsection{quizReducer.js} % (fold)
\label{ssub:quizreducer_js}
\lstinputlisting[language=javascript, caption=Quiz reducer controlling creation state.]{../src/reducers/quizReducer.js}
% subsubsection quizreducer_js (end)

\subsubsection{currentQuizReducer.js} % (fold)
\label{ssub:currentquizreducer_js}
\lstinputlisting[language=javascript, caption=Reducer controlling the current state for a quiz that is being played.]{../src/reducers/currentQuizReducer.js}
% subsubsection currentquizreducer_js (end)

% subsection reducers (end)

\subsection{Components} % (fold)
\label{sub:reducers}
This subsections lists the application's components. Components make up the vast majority of the codebase, as they combine both markup - a description of what should be shown - and logic. The first three components are more advanced, as they interact directly with the applications's state. Other components are less intelligent, and rely on properties passed down from the more advanced components to perform actions and read data.
\subsubsection{LoginContainer.js} % (fold)
\label{ssub:login_js}
\lstinputlisting[language=javascript, caption=Advanced component controlling the login screen.]{../src/containers/LoginContainer.js}
% subsubsection login_js (end)

\subsubsection{LoginForm.js} % (fold)
\label{ssub:loginform_js}
\lstinputlisting[language=javascript, caption=Login form component.]{../src/components/LoginForm.js}
% subsubsection loginform_js (end)

\subsubsection{PlayQuizContainer.js} % (fold)
\label{ssub:playquiz_js}
\lstinputlisting[language=javascript, caption=Advanced component controlling the play quiz screen.]{../src/containers/PlayQuizContainer.js}
% subsubsection playquiz_js (end)

\subsubsection{PlayQuiz.js} % (fold)
\label{ssub:loginform_js}
\lstinputlisting[language=javascript, caption=Play quiz component.]{../src/components/play/PlayQuiz.js}
% subsubsection loginform_js (end)

\subsubsection{Countdown.js} % (fold)
\label{ssub:countdown_js}
\lstinputlisting[language=javascript, caption=Displays the time until the quiz is due to start.]{../src/components/play/Countdown.js}
% subsubsection countdown_js (end)

\subsubsection{QuestionTimer.js} % (fold)
\label{ssub:questiontimer_js}
\lstinputlisting[language=javascript, caption=Counts down the time until the next question.]{../src/components/play/QuestionTimer.js}
% subsubsection questiontimer_js (end)

\subsubsection{CreateQuizContainer.js} % (fold)
\label{ssub:createquiz_js}
\lstinputlisting[language=javascript, caption=Advanced component controlling the create quiz screen.]{../src/containers/CreateQuizContainer.js}
% subsubsection createquiz_js (end)

\subsubsection{CreateQuiz.js} % (fold)
\label{ssub:createquiz_js}
\lstinputlisting[language=javascript, caption=Create quiz component]{../src/components/CreateQuiz.js}
% subsubsection createquiz_js (end)

\subsubsection{Button.js} % (fold)
\label{ssub:yearselector_js}
\lstinputlisting[language=javascript, caption=Button box component.]{../src/components/Button.js}
% subsubsection yearselector_js (end)

\subsubsection{Select.js} % (fold)
\label{ssub:yearselector_js}
\lstinputlisting[language=javascript, caption=Select box component.]{../src/components/Select.js}
% subsubsection yearselector_js (end)

\subsubsection{Answer.js} % (fold)
\label{ssub:answer_js}
\lstinputlisting[language=javascript, caption=Quiz creator answer component.]{../src/components/Answer.js}
% subsubsection answer_js (end)

\subsubsection{EditableText.js} % (fold)
\label{ssub:editabletext_js}
\lstinputlisting[language=javascript, caption=Dual heading / input component.]{../src/components/EditableText.js}
% subsubsection editabletext_js (end)

\subsubsection{QuizSettingsPanel.js} % (fold)
\label{ssub:quizsettingspanel_js}
\lstinputlisting[language=javascript, caption=Quiz settings panel.]{../src/components/QuizSettingsPanel.js}
% subsubsection quizsettingspanel_js (end)

% subsection reducers (end)

\subsection{Application Tooling} % (fold)
\label{sub:api_models}
This subsection lists the tooling and build process that is required for the application to run. Though not technically part of the source, a strong understanding is vital if one is to come to terms with the project.
\subsubsection{index.js} % (fold)
\label{ssub:index_js}
\lstinputlisting[language=html, caption=Main application entry.]{../src/index.js}
% subsubsection index_js (end)

\subsubsection{index.html} % (fold)
This file serves as an entry point for the client.
\lstinputlisting[language=html, caption=Main HTML entry.]{../index.html}
% subsubsection server_js (end)

\subsubsection{webpack.config.babel.js} % (fold)
This file compiles all the client dependencies for the application.
\lstinputlisting[language=javascript, caption=Bootstraps the backend.]{../webpack.config.js}
% subsubsection server_js (end)

% subsection api_models (end)

\subsection{Styles} % (fold)
\label{sub:styles}
This subsection lists the stylesheets that control the majority of the application's visual aspects.
\subsubsection{porfiry.scss} % (fold)
\label{ssub:porfiry_scss}
\lstinputlisting[language=css, caption=Entry point for styles and for importing the other sheets.]{../src/styles/porfiry.scss}
% subsubsection porfiry_scss (end)

\subsubsection{typography.scss} % (fold)
\label{ssub:_typography_scss}
\lstinputlisting[language=css, caption=Styling for typographical elements of the application.]{../src/styles/base/_typography.scss}
% subsubsection _typography_scss (end)

\subsubsection{answer.scss} % (fold)
\label{ssub:answer_scss}
\lstinputlisting[language=css, caption=Styling for the answer blocks.]{../src/styles/components/_answer.scss}
% subsubsection answer_scss (end)

\subsubsection{button.scss} % (fold)
\label{ssub:button_scss}
\lstinputlisting[language=css, caption=Styling for the buttons.]{../src/styles/components/_button.scss}
% subsubsection button_scss (end)

\subsubsection{input.scss} % (fold)
\label{ssub:input_scss}
\lstinputlisting[language=css, caption=Styling for the text input elements.]{../src/styles/components/_input.scss}
% subsubsection input_scss (end)

\subsubsection{notification.scss} % (fold)
\label{ssub:notifications_scss}
\lstinputlisting[language=css, caption=Styling for the alert notifications.]{../src/styles/components/_notifications.scss}
% subsubsection notifications_scss (end)

\subsubsection{select.scss} % (fold)
\label{ssub:select_scss}
\lstinputlisting[language=css, caption=Styling for the select elements.]{../src/styles/components/_select.scss}
% subsubsection select_scss (end)

\subsubsection{login.scss} % (fold)
\label{ssub:login_scss}
\lstinputlisting[language=css, caption=Styling for the login form ensuring it is positioned properly.]{../src/styles/containers/login.scss}
% subsubsection login_scss (end)

\subsubsection{variables.scss} % (fold)
\label{ssub:variables_scss}
\lstinputlisting[language=css, caption=Variables used throughout the styles.]{../src/styles/helpers/_variables.scss}
% subsubsection variables_scss (end)

\subsubsection{mixins.scss} % (fold)
\label{ssub:mixins_scss}
\lstinputlisting[language=css, caption=Mixins used to apply processes across styles.]{../src/styles/helpers/_mixins.scss}
% subsubsection mixins_scss (end)

% subsection styles (end)
