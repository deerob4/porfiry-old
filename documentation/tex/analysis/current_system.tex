\section{Investigation of the Current System}

\subsection{Overview of System}

\subsection{Interviews}

\subsection{Questionnaires}

\subsection{Observations}
Having been a member of the school community for over five years, I am well placed to provide an observation on how the school currently goes about creating, setting and analysing quizzes used in form times. The current process of creating a quiz follows this regular pattern:

\begin{enumerate}
\item The head of year creating the quiz thinks of a set of questions and possible answers, usually following a theme, and then writes them down on a Microsoft Word document. The correct answer is marked out, to aid the form tutor in marking the quiz. This document is then saved to a drive on the school's LAN.

\item The head of year then notifies the individual form tutors of the quiz, usually at one of their weekly meetings, and tells them to conduct the quiz with their form group on a certain date.

\item When the date is reached, the form tutor opens the document from the network, ensuring that the document is kept hidden to avoid members of the form viewing the correct answers.

\item The form tutor reads each question out in turn, and the members of the form work together to attempt to work out the answer. They either come up with their own answer or choose from a list of options, depending on whether or not the choice is multiple choice. The form tutor marks down the answer they chose, and this process repeats until the quiz is completed.

\item Once all the questions have been answered, the form tutor adds up the total number of marks achieved by the form.

\item The form tutor then passes the mark onto the head of year, either by email or when passing them in the corridor or the staffroom. This task is sometimes performed by a member of the form themselves, occasionally with the expectation that the mark achieved will be exaggerated somewhat.

\item After receiving all the results, the head of year works which form achieved the highest result, and which the lowest. This result is reported back to the year in the weekly assembly, often with a small reward for the highest achieving form.
\end{enumerate}

There are evidently a large number of issues with the above method. Firstly, distributing the quizzes via a word-processed document is not a particularly efficient method. 

\subsection{Document Inspections}

\subsection{Similar Systems}
There are a number of systems available, both free and at a cost, that would allow the school to improve their current method of quiztribution (\textit{quiz distribution}). Several popular options are outlined below.

\subsubsection{Quiz Creation Websites}
A numer of websites exist that allow users to design, play and share their own quizzes. These websites, including \textit{QuizWorks}, \textit{ExamTime} and \textit{QuizBean} generally follow the same pattern: the user creates an account, is directed to an interface wherein they can design a quiz, and is then given a link with which they can share the quiz with others. For basic quiz creation, these websites are free, though for more advanced usage (\textit{QuizWorks} defines an ``advanced'' quiz as one containing more than 15 questions), paid plans are available.

As these systems are websites, they can be accessed from practically any computer or mobile device, as long as there is an internet connection in range. This means that users can continue to work on their quizzes, whether designing or answering them, outside of their place of work.

Though these systems are undoubtedly useful, and could, with a few compromises, be easily integrated into the school's routines, they lack an awareness of the structure of a school. There is no concept of ``form groups'' or ``heads of year'', both are which are vital concepts if the system is to meet what the school desires. Additionally, they lack the ability to display a detailed analysis of the results (at least, not without paying a somewhat exhorbitant fee - \pounds60 per month in the case of \textit{QuizWorks}), a side effect of their focus on individuals as opposed to groups.

\subsubsection{Quiz Creation Software Packages}
Similar to quiz creation websites, quiz creation software packages allow the user to design and play a quiz. However, these systems are desktop applications (the majority are designed for Microsoft Windows), and so can only be accessed from a single desktop or laptop system. Examples of these systems include \textit{Wondershare Quiz Creator}, \textit{Tanida QuizBuilder}, and \textit{Articulate Storyline 2}. Unlike the mostly free websites, these software packages are often very expensive: the three systems mentioned range in price from \$99 - \$1846 for a single license, with additional licenses costing even more.

To compensate for the high prices, these desktop applications contain a vast feature set. Quizzes of every imaginable type can be created, from drag-and-drop, multiple choice, word bank quizzes, and many more. Images can be included, points assigned, and complex animations can be set to make the quiz as visually appealing as possible. In addition, reports can be generated with tremendous amounts of data, showcasing practically every data point imaginable.

Useful though these features are, 

\subsubsection{Quizdom}


\subsection{Justification of Methods}

\subsection{IPSO Chart}

\subsection{Limitations of Current System}
