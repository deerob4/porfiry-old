\section{Evaluation Critera}
When evaluating the system, several factors will to be taken into account to determine if the system itself, and the development process, is a success. They are listed below:

\begin{itemize}
\item The system should be able to cope with at least 1000 connections at any one time, all communicating with each other. This communication should all be real time and seamless, and there should not be any lags or stutters caused directly by the system.
\item Staff and students should find the user interface simple and easy to use; they should instinctively know how to use the system, with a minimum amount of training required.
\item As the school requires the system relatively quickly, it is important that development of the system is completed within a reasonable period of time. To achieve this, it is important that time is spent only on features that the school have asked for, and not on ``useful'' extras.
\item The application code should be as high quality as possible - where possible, functional paradigms, such as using \textit{map} as opposed to a loop, should be followed, and code should be kept as modular and reusable as possible, to bring in all the advantages that such a development model provides.
\item The cost of the system to run should also be taken into account. Though the school will not be charged directly for the system, there will be costs relating to its continued maintenance and uptime. If these are overly expensive, there will be a negative impact on the success of the project.
\item Stability is an important factor, and it will also be considered. If the system is constantly prone to crashing, it could not be called stable; likewise, it would also indicate a failure of the test plan.
\item The suitability of the test plan will also be evaluated. A test plan is important because it provides a way of ensuring the parts of the system function correctly.
\end{itemize}

If, after the completion of the project, these criteria are not met, it would be impossible, or at least very difficult, to label the system a complete success.
