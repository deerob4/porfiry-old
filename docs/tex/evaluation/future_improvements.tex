\section{Future Improvements} % (fold)
\label{sec:future_improvements}
% Add commentarties to test results

There are a tremendous number of ways in which the system can be improved. Perhaps the most obvious is fully implementing the live visualisation system, whereby the current status of each house and form is relayed in real time to each individual user. As set out in the evaluation above, there are a number of factors that make this relatively difficult to achieve, but it is definitely possible.

One possible feature could be the ability to see if a form or house has consistently won over several quizzes. This way, a running tally could be kept of the best form or house, increasing competitiveness between houses, one of the system's inital goals. This could be taken further with a streak system: for every \textit{n} number of quizzes the house has won, they could receive \textit{n} extra house points. Such as system would also allow individual forms that have done especially well to be reported. Of course, such an approach could make it difficult for other houses to catch up and beat the house with the highest streak: by winning just a few quizzes at first, their score would eventually snowball, placing them at an advantage during subsequent quizzes.

To solve this, a ``bonus'' system could be implemented. Every so often, special events, such as a knockout question between just two individual students, could be held, worth a large number of house points. These could take place at random intervals, perhaps between questions or categories, and would likely help to increase engagement with the quiz, an issue that was raised in the interviews. By giving out a large number of points every so often, the ``incumbent'' house with a large streak could be unfooted, making the system fairer.

A further change that might be considered is rewriting the application's structure and backend. The system currently utilises a fairly loose structure, with little coordination between files and directories. It would be beneficial to reimplement the backend in a language that enforces such a structure. A potential candidate for this might be Erlang, a language well suited for building highly concurrent, real-time applications, much like the quiz system. Roughly 40\% of the world's telephone and SMS communications pass through Erlang powererd servers, giving it a well deserved reputation of stability. The langauge also more rigidily enforces a functional style of programming. Though a, largely successful, drive to stick to the paradigm was a key criteria in the current version of the system, there are still certain areas where the code more resembles that seen in an imperative or object-oriented system, or where the code could simply be refactored to be much simpler. Using a more functional language would help improve this.

Other useful features could also be included, such as a backup feature to ensure that quizzes are never lost. Currently they are stored in only the one database, with no way to export them. Since they can easily be represented as simple JSON, it would be a simple case to include an export button, allowing the school to save their most popular quizzes to their own hard drive or storage area, ensuring that they are never lost. This would also make it easier for the school to reuse certain aspects of quizzes

This could be further improved by including a ``template'' feature. This way, quizzes could be made from other quizzes, containing a preset set of questions; this would allowing popular quizzes to be extended upon without having to start from scratch each time. The system already includes functionality to facilitate this, however: once a quiz has been played by students, it is possible to just edit it and reschedule it for a different time, so the school could simply use this method.

It could also be an idea to delete user data every five years automatically, ensuring that data only lasts as long as students are in the school. Otherwise, any streaks that are gained through the above improvement could be the result of ex-students from several years ago, and not an accurate reflection of the current student body.

It is also a little time consuming to come up with questions and answers every time. It would be helpful to have a bank of predefined questions and potential answers integrated into the creator interface, allowing the school to simply select the categories of questions they want, and have the quiz be automatically generated. This would substantially increase development time, however, as a wide and suitable range of questions and answers would have to be written across a number of categories.
% section future_improvements (end)
