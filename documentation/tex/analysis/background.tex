\section{Background}

\subsection{About the Business}

The Priory School is a medium sized secondary school located in Shrewsbury, Shropshire. The school is a founding member of the Salop Teaching Alliance, and employs over 100 teaching staff, with approximately 900 pupils on roll. Pupils range in age from 11 - 16, and each belongs to an individual form group. During it's previous two inspections, Ofsted judged the school to be Outstanding, the highest possible rating. Additionally, the school has the highest attendance rate in the county, and achieves exam results well above the national average. 

A change of leadership in January 2015 resulted in the previous headteacher, Ms Candy Garbett, leaving the school; Mr Michael Barratt, previously of Adams Grammar School, Newport, became the new principal. Following this change in leadership, the school has sought to embrace the advantages of technology, and has invested in several new systems, including an online homework tracker, a virtual learning environment, and a library tracking system. This newfound acceptance of technology opens the way for this project.

Due to Michael Barratt's previous experience in a grammar school, a number of additional changes have taken place, particularly relating to how students are organised throughout the school. Previously, each student belonged to a form group - one of Acton, Baxter, Clive, Houseman, Darwin or Webb. There were five of these houses throughout the school, one for each year - a 7A, 8A, 9A, 10A and 11A.

 Each year was controlled by a senior member of staff, usually one who had worked at the school for a large period of time; these were known as the \textit{head of year}. For example, the head of Year 7, Nick Bucknall, would be in charge of 7A, 7B, 7C, 7H, 7D and 7W, and would meet with the Year 7 form tutors each week, provide assemblies for Year 7, and so on. 

Owing to Barrat's fondness for the more traditional house system, this horizontal approach has now been modified to a vertical one: each previous head of year is now known as a \textit{head of house}; and, rather than controlling an individual year group, they now control an individual house. For example, Nick Bucknall has been placed in charge of Acton, and so controls 7A - 11A.

\subsection{About the Project}
Like many schools, The Priory School makes use of a form time in the afternoon. During this process, students are registered, bulletins are read out, and a timetabled activity is carried out; these activities usually include silent reading, a group debate, and quizzes. These quizzes are usually designed by the head of year, and include a range of topics, from current affairs, educational matters, and simple trivia.

Currently, these quizzes are delivered to the forms on a Microsoft Word document, via the school's LAN. The members of the form work together to arrive at what they believe to be the correct answer, and once all questions have been answered, the form tutor marks the quiz and returns the result to the head of year, usually orally. Which would aid in following the school's new policy of ``togetherness''.